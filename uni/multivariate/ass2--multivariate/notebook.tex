
% Default to the notebook output style

    


% Inherit from the specified cell style.




    
\documentclass[11pt]{article}

    
    
    \usepackage[T1]{fontenc}
    % Nicer default font (+ math font) than Computer Modern for most use cases
    \usepackage{mathpazo}

    % Basic figure setup, for now with no caption control since it's done
    % automatically by Pandoc (which extracts ![](path) syntax from Markdown).
    \usepackage{graphicx}
    % We will generate all images so they have a width \maxwidth. This means
    % that they will get their normal width if they fit onto the page, but
    % are scaled down if they would overflow the margins.
    \makeatletter
    \def\maxwidth{\ifdim\Gin@nat@width>\linewidth\linewidth
    \else\Gin@nat@width\fi}
    \makeatother
    \let\Oldincludegraphics\includegraphics
    % Set max figure width to be 80% of text width, for now hardcoded.
    \renewcommand{\includegraphics}[1]{\Oldincludegraphics[width=.8\maxwidth]{#1}}
    % Ensure that by default, figures have no caption (until we provide a
    % proper Figure object with a Caption API and a way to capture that
    % in the conversion process - todo).
    \usepackage{caption}
    \DeclareCaptionLabelFormat{nolabel}{}
    \captionsetup{labelformat=nolabel}

    \usepackage{adjustbox} % Used to constrain images to a maximum size 
    \usepackage{xcolor} % Allow colors to be defined
    \usepackage{enumerate} % Needed for markdown enumerations to work
    \usepackage{geometry} % Used to adjust the document margins
    \usepackage{amsmath} % Equations
    \usepackage{amssymb} % Equations
    \usepackage{textcomp} % defines textquotesingle
    % Hack from http://tex.stackexchange.com/a/47451/13684:
    \AtBeginDocument{%
        \def\PYZsq{\textquotesingle}% Upright quotes in Pygmentized code
    }
    \usepackage{upquote} % Upright quotes for verbatim code
    \usepackage{eurosym} % defines \euro
    \usepackage[mathletters]{ucs} % Extended unicode (utf-8) support
    \usepackage[utf8x]{inputenc} % Allow utf-8 characters in the tex document
    \usepackage{fancyvrb} % verbatim replacement that allows latex
    \usepackage{grffile} % extends the file name processing of package graphics 
                         % to support a larger range 
    % The hyperref package gives us a pdf with properly built
    % internal navigation ('pdf bookmarks' for the table of contents,
    % internal cross-reference links, web links for URLs, etc.)
    \usepackage{hyperref}
    \usepackage{longtable} % longtable support required by pandoc >1.10
    \usepackage{booktabs}  % table support for pandoc > 1.12.2
    \usepackage[inline]{enumitem} % IRkernel/repr support (it uses the enumerate* environment)
    \usepackage[normalem]{ulem} % ulem is needed to support strikethroughs (\sout)
                                % normalem makes italics be italics, not underlines
    

    
    
    % Colors for the hyperref package
    \definecolor{urlcolor}{rgb}{0,.145,.698}
    \definecolor{linkcolor}{rgb}{.71,0.21,0.01}
    \definecolor{citecolor}{rgb}{.12,.54,.11}

    % ANSI colors
    \definecolor{ansi-black}{HTML}{3E424D}
    \definecolor{ansi-black-intense}{HTML}{282C36}
    \definecolor{ansi-red}{HTML}{E75C58}
    \definecolor{ansi-red-intense}{HTML}{B22B31}
    \definecolor{ansi-green}{HTML}{00A250}
    \definecolor{ansi-green-intense}{HTML}{007427}
    \definecolor{ansi-yellow}{HTML}{DDB62B}
    \definecolor{ansi-yellow-intense}{HTML}{B27D12}
    \definecolor{ansi-blue}{HTML}{208FFB}
    \definecolor{ansi-blue-intense}{HTML}{0065CA}
    \definecolor{ansi-magenta}{HTML}{D160C4}
    \definecolor{ansi-magenta-intense}{HTML}{A03196}
    \definecolor{ansi-cyan}{HTML}{60C6C8}
    \definecolor{ansi-cyan-intense}{HTML}{258F8F}
    \definecolor{ansi-white}{HTML}{C5C1B4}
    \definecolor{ansi-white-intense}{HTML}{A1A6B2}

    % commands and environments needed by pandoc snippets
    % extracted from the output of `pandoc -s`
    \providecommand{\tightlist}{%
      \setlength{\itemsep}{0pt}\setlength{\parskip}{0pt}}
    \DefineVerbatimEnvironment{Highlighting}{Verbatim}{commandchars=\\\{\}}
    % Add ',fontsize=\small' for more characters per line
    \newenvironment{Shaded}{}{}
    \newcommand{\KeywordTok}[1]{\textcolor[rgb]{0.00,0.44,0.13}{\textbf{{#1}}}}
    \newcommand{\DataTypeTok}[1]{\textcolor[rgb]{0.56,0.13,0.00}{{#1}}}
    \newcommand{\DecValTok}[1]{\textcolor[rgb]{0.25,0.63,0.44}{{#1}}}
    \newcommand{\BaseNTok}[1]{\textcolor[rgb]{0.25,0.63,0.44}{{#1}}}
    \newcommand{\FloatTok}[1]{\textcolor[rgb]{0.25,0.63,0.44}{{#1}}}
    \newcommand{\CharTok}[1]{\textcolor[rgb]{0.25,0.44,0.63}{{#1}}}
    \newcommand{\StringTok}[1]{\textcolor[rgb]{0.25,0.44,0.63}{{#1}}}
    \newcommand{\CommentTok}[1]{\textcolor[rgb]{0.38,0.63,0.69}{\textit{{#1}}}}
    \newcommand{\OtherTok}[1]{\textcolor[rgb]{0.00,0.44,0.13}{{#1}}}
    \newcommand{\AlertTok}[1]{\textcolor[rgb]{1.00,0.00,0.00}{\textbf{{#1}}}}
    \newcommand{\FunctionTok}[1]{\textcolor[rgb]{0.02,0.16,0.49}{{#1}}}
    \newcommand{\RegionMarkerTok}[1]{{#1}}
    \newcommand{\ErrorTok}[1]{\textcolor[rgb]{1.00,0.00,0.00}{\textbf{{#1}}}}
    \newcommand{\NormalTok}[1]{{#1}}
    
    % Additional commands for more recent versions of Pandoc
    \newcommand{\ConstantTok}[1]{\textcolor[rgb]{0.53,0.00,0.00}{{#1}}}
    \newcommand{\SpecialCharTok}[1]{\textcolor[rgb]{0.25,0.44,0.63}{{#1}}}
    \newcommand{\VerbatimStringTok}[1]{\textcolor[rgb]{0.25,0.44,0.63}{{#1}}}
    \newcommand{\SpecialStringTok}[1]{\textcolor[rgb]{0.73,0.40,0.53}{{#1}}}
    \newcommand{\ImportTok}[1]{{#1}}
    \newcommand{\DocumentationTok}[1]{\textcolor[rgb]{0.73,0.13,0.13}{\textit{{#1}}}}
    \newcommand{\AnnotationTok}[1]{\textcolor[rgb]{0.38,0.63,0.69}{\textbf{\textit{{#1}}}}}
    \newcommand{\CommentVarTok}[1]{\textcolor[rgb]{0.38,0.63,0.69}{\textbf{\textit{{#1}}}}}
    \newcommand{\VariableTok}[1]{\textcolor[rgb]{0.10,0.09,0.49}{{#1}}}
    \newcommand{\ControlFlowTok}[1]{\textcolor[rgb]{0.00,0.44,0.13}{\textbf{{#1}}}}
    \newcommand{\OperatorTok}[1]{\textcolor[rgb]{0.40,0.40,0.40}{{#1}}}
    \newcommand{\BuiltInTok}[1]{{#1}}
    \newcommand{\ExtensionTok}[1]{{#1}}
    \newcommand{\PreprocessorTok}[1]{\textcolor[rgb]{0.74,0.48,0.00}{{#1}}}
    \newcommand{\AttributeTok}[1]{\textcolor[rgb]{0.49,0.56,0.16}{{#1}}}
    \newcommand{\InformationTok}[1]{\textcolor[rgb]{0.38,0.63,0.69}{\textbf{\textit{{#1}}}}}
    \newcommand{\WarningTok}[1]{\textcolor[rgb]{0.38,0.63,0.69}{\textbf{\textit{{#1}}}}}
    
    
    % Define a nice break command that doesn't care if a line doesn't already
    % exist.
    \def\br{\hspace*{\fill} \\* }
    % Math Jax compatability definitions
    \def\gt{>}
    \def\lt{<}
    % Document parameters
    \title{ass2-multivariate}
    
    
    

    % Pygments definitions
    
\makeatletter
\def\PY@reset{\let\PY@it=\relax \let\PY@bf=\relax%
    \let\PY@ul=\relax \let\PY@tc=\relax%
    \let\PY@bc=\relax \let\PY@ff=\relax}
\def\PY@tok#1{\csname PY@tok@#1\endcsname}
\def\PY@toks#1+{\ifx\relax#1\empty\else%
    \PY@tok{#1}\expandafter\PY@toks\fi}
\def\PY@do#1{\PY@bc{\PY@tc{\PY@ul{%
    \PY@it{\PY@bf{\PY@ff{#1}}}}}}}
\def\PY#1#2{\PY@reset\PY@toks#1+\relax+\PY@do{#2}}

\expandafter\def\csname PY@tok@w\endcsname{\def\PY@tc##1{\textcolor[rgb]{0.73,0.73,0.73}{##1}}}
\expandafter\def\csname PY@tok@c\endcsname{\let\PY@it=\textit\def\PY@tc##1{\textcolor[rgb]{0.25,0.50,0.50}{##1}}}
\expandafter\def\csname PY@tok@cp\endcsname{\def\PY@tc##1{\textcolor[rgb]{0.74,0.48,0.00}{##1}}}
\expandafter\def\csname PY@tok@k\endcsname{\let\PY@bf=\textbf\def\PY@tc##1{\textcolor[rgb]{0.00,0.50,0.00}{##1}}}
\expandafter\def\csname PY@tok@kp\endcsname{\def\PY@tc##1{\textcolor[rgb]{0.00,0.50,0.00}{##1}}}
\expandafter\def\csname PY@tok@kt\endcsname{\def\PY@tc##1{\textcolor[rgb]{0.69,0.00,0.25}{##1}}}
\expandafter\def\csname PY@tok@o\endcsname{\def\PY@tc##1{\textcolor[rgb]{0.40,0.40,0.40}{##1}}}
\expandafter\def\csname PY@tok@ow\endcsname{\let\PY@bf=\textbf\def\PY@tc##1{\textcolor[rgb]{0.67,0.13,1.00}{##1}}}
\expandafter\def\csname PY@tok@nb\endcsname{\def\PY@tc##1{\textcolor[rgb]{0.00,0.50,0.00}{##1}}}
\expandafter\def\csname PY@tok@nf\endcsname{\def\PY@tc##1{\textcolor[rgb]{0.00,0.00,1.00}{##1}}}
\expandafter\def\csname PY@tok@nc\endcsname{\let\PY@bf=\textbf\def\PY@tc##1{\textcolor[rgb]{0.00,0.00,1.00}{##1}}}
\expandafter\def\csname PY@tok@nn\endcsname{\let\PY@bf=\textbf\def\PY@tc##1{\textcolor[rgb]{0.00,0.00,1.00}{##1}}}
\expandafter\def\csname PY@tok@ne\endcsname{\let\PY@bf=\textbf\def\PY@tc##1{\textcolor[rgb]{0.82,0.25,0.23}{##1}}}
\expandafter\def\csname PY@tok@nv\endcsname{\def\PY@tc##1{\textcolor[rgb]{0.10,0.09,0.49}{##1}}}
\expandafter\def\csname PY@tok@no\endcsname{\def\PY@tc##1{\textcolor[rgb]{0.53,0.00,0.00}{##1}}}
\expandafter\def\csname PY@tok@nl\endcsname{\def\PY@tc##1{\textcolor[rgb]{0.63,0.63,0.00}{##1}}}
\expandafter\def\csname PY@tok@ni\endcsname{\let\PY@bf=\textbf\def\PY@tc##1{\textcolor[rgb]{0.60,0.60,0.60}{##1}}}
\expandafter\def\csname PY@tok@na\endcsname{\def\PY@tc##1{\textcolor[rgb]{0.49,0.56,0.16}{##1}}}
\expandafter\def\csname PY@tok@nt\endcsname{\let\PY@bf=\textbf\def\PY@tc##1{\textcolor[rgb]{0.00,0.50,0.00}{##1}}}
\expandafter\def\csname PY@tok@nd\endcsname{\def\PY@tc##1{\textcolor[rgb]{0.67,0.13,1.00}{##1}}}
\expandafter\def\csname PY@tok@s\endcsname{\def\PY@tc##1{\textcolor[rgb]{0.73,0.13,0.13}{##1}}}
\expandafter\def\csname PY@tok@sd\endcsname{\let\PY@it=\textit\def\PY@tc##1{\textcolor[rgb]{0.73,0.13,0.13}{##1}}}
\expandafter\def\csname PY@tok@si\endcsname{\let\PY@bf=\textbf\def\PY@tc##1{\textcolor[rgb]{0.73,0.40,0.53}{##1}}}
\expandafter\def\csname PY@tok@se\endcsname{\let\PY@bf=\textbf\def\PY@tc##1{\textcolor[rgb]{0.73,0.40,0.13}{##1}}}
\expandafter\def\csname PY@tok@sr\endcsname{\def\PY@tc##1{\textcolor[rgb]{0.73,0.40,0.53}{##1}}}
\expandafter\def\csname PY@tok@ss\endcsname{\def\PY@tc##1{\textcolor[rgb]{0.10,0.09,0.49}{##1}}}
\expandafter\def\csname PY@tok@sx\endcsname{\def\PY@tc##1{\textcolor[rgb]{0.00,0.50,0.00}{##1}}}
\expandafter\def\csname PY@tok@m\endcsname{\def\PY@tc##1{\textcolor[rgb]{0.40,0.40,0.40}{##1}}}
\expandafter\def\csname PY@tok@gh\endcsname{\let\PY@bf=\textbf\def\PY@tc##1{\textcolor[rgb]{0.00,0.00,0.50}{##1}}}
\expandafter\def\csname PY@tok@gu\endcsname{\let\PY@bf=\textbf\def\PY@tc##1{\textcolor[rgb]{0.50,0.00,0.50}{##1}}}
\expandafter\def\csname PY@tok@gd\endcsname{\def\PY@tc##1{\textcolor[rgb]{0.63,0.00,0.00}{##1}}}
\expandafter\def\csname PY@tok@gi\endcsname{\def\PY@tc##1{\textcolor[rgb]{0.00,0.63,0.00}{##1}}}
\expandafter\def\csname PY@tok@gr\endcsname{\def\PY@tc##1{\textcolor[rgb]{1.00,0.00,0.00}{##1}}}
\expandafter\def\csname PY@tok@ge\endcsname{\let\PY@it=\textit}
\expandafter\def\csname PY@tok@gs\endcsname{\let\PY@bf=\textbf}
\expandafter\def\csname PY@tok@gp\endcsname{\let\PY@bf=\textbf\def\PY@tc##1{\textcolor[rgb]{0.00,0.00,0.50}{##1}}}
\expandafter\def\csname PY@tok@go\endcsname{\def\PY@tc##1{\textcolor[rgb]{0.53,0.53,0.53}{##1}}}
\expandafter\def\csname PY@tok@gt\endcsname{\def\PY@tc##1{\textcolor[rgb]{0.00,0.27,0.87}{##1}}}
\expandafter\def\csname PY@tok@err\endcsname{\def\PY@bc##1{\setlength{\fboxsep}{0pt}\fcolorbox[rgb]{1.00,0.00,0.00}{1,1,1}{\strut ##1}}}
\expandafter\def\csname PY@tok@kc\endcsname{\let\PY@bf=\textbf\def\PY@tc##1{\textcolor[rgb]{0.00,0.50,0.00}{##1}}}
\expandafter\def\csname PY@tok@kd\endcsname{\let\PY@bf=\textbf\def\PY@tc##1{\textcolor[rgb]{0.00,0.50,0.00}{##1}}}
\expandafter\def\csname PY@tok@kn\endcsname{\let\PY@bf=\textbf\def\PY@tc##1{\textcolor[rgb]{0.00,0.50,0.00}{##1}}}
\expandafter\def\csname PY@tok@kr\endcsname{\let\PY@bf=\textbf\def\PY@tc##1{\textcolor[rgb]{0.00,0.50,0.00}{##1}}}
\expandafter\def\csname PY@tok@bp\endcsname{\def\PY@tc##1{\textcolor[rgb]{0.00,0.50,0.00}{##1}}}
\expandafter\def\csname PY@tok@fm\endcsname{\def\PY@tc##1{\textcolor[rgb]{0.00,0.00,1.00}{##1}}}
\expandafter\def\csname PY@tok@vc\endcsname{\def\PY@tc##1{\textcolor[rgb]{0.10,0.09,0.49}{##1}}}
\expandafter\def\csname PY@tok@vg\endcsname{\def\PY@tc##1{\textcolor[rgb]{0.10,0.09,0.49}{##1}}}
\expandafter\def\csname PY@tok@vi\endcsname{\def\PY@tc##1{\textcolor[rgb]{0.10,0.09,0.49}{##1}}}
\expandafter\def\csname PY@tok@vm\endcsname{\def\PY@tc##1{\textcolor[rgb]{0.10,0.09,0.49}{##1}}}
\expandafter\def\csname PY@tok@sa\endcsname{\def\PY@tc##1{\textcolor[rgb]{0.73,0.13,0.13}{##1}}}
\expandafter\def\csname PY@tok@sb\endcsname{\def\PY@tc##1{\textcolor[rgb]{0.73,0.13,0.13}{##1}}}
\expandafter\def\csname PY@tok@sc\endcsname{\def\PY@tc##1{\textcolor[rgb]{0.73,0.13,0.13}{##1}}}
\expandafter\def\csname PY@tok@dl\endcsname{\def\PY@tc##1{\textcolor[rgb]{0.73,0.13,0.13}{##1}}}
\expandafter\def\csname PY@tok@s2\endcsname{\def\PY@tc##1{\textcolor[rgb]{0.73,0.13,0.13}{##1}}}
\expandafter\def\csname PY@tok@sh\endcsname{\def\PY@tc##1{\textcolor[rgb]{0.73,0.13,0.13}{##1}}}
\expandafter\def\csname PY@tok@s1\endcsname{\def\PY@tc##1{\textcolor[rgb]{0.73,0.13,0.13}{##1}}}
\expandafter\def\csname PY@tok@mb\endcsname{\def\PY@tc##1{\textcolor[rgb]{0.40,0.40,0.40}{##1}}}
\expandafter\def\csname PY@tok@mf\endcsname{\def\PY@tc##1{\textcolor[rgb]{0.40,0.40,0.40}{##1}}}
\expandafter\def\csname PY@tok@mh\endcsname{\def\PY@tc##1{\textcolor[rgb]{0.40,0.40,0.40}{##1}}}
\expandafter\def\csname PY@tok@mi\endcsname{\def\PY@tc##1{\textcolor[rgb]{0.40,0.40,0.40}{##1}}}
\expandafter\def\csname PY@tok@il\endcsname{\def\PY@tc##1{\textcolor[rgb]{0.40,0.40,0.40}{##1}}}
\expandafter\def\csname PY@tok@mo\endcsname{\def\PY@tc##1{\textcolor[rgb]{0.40,0.40,0.40}{##1}}}
\expandafter\def\csname PY@tok@ch\endcsname{\let\PY@it=\textit\def\PY@tc##1{\textcolor[rgb]{0.25,0.50,0.50}{##1}}}
\expandafter\def\csname PY@tok@cm\endcsname{\let\PY@it=\textit\def\PY@tc##1{\textcolor[rgb]{0.25,0.50,0.50}{##1}}}
\expandafter\def\csname PY@tok@cpf\endcsname{\let\PY@it=\textit\def\PY@tc##1{\textcolor[rgb]{0.25,0.50,0.50}{##1}}}
\expandafter\def\csname PY@tok@c1\endcsname{\let\PY@it=\textit\def\PY@tc##1{\textcolor[rgb]{0.25,0.50,0.50}{##1}}}
\expandafter\def\csname PY@tok@cs\endcsname{\let\PY@it=\textit\def\PY@tc##1{\textcolor[rgb]{0.25,0.50,0.50}{##1}}}

\def\PYZbs{\char`\\}
\def\PYZus{\char`\_}
\def\PYZob{\char`\{}
\def\PYZcb{\char`\}}
\def\PYZca{\char`\^}
\def\PYZam{\char`\&}
\def\PYZlt{\char`\<}
\def\PYZgt{\char`\>}
\def\PYZsh{\char`\#}
\def\PYZpc{\char`\%}
\def\PYZdl{\char`\$}
\def\PYZhy{\char`\-}
\def\PYZsq{\char`\'}
\def\PYZdq{\char`\"}
\def\PYZti{\char`\~}
% for compatibility with earlier versions
\def\PYZat{@}
\def\PYZlb{[}
\def\PYZrb{]}
\makeatother


    % Exact colors from NB
    \definecolor{incolor}{rgb}{0.0, 0.0, 0.5}
    \definecolor{outcolor}{rgb}{0.545, 0.0, 0.0}



    
    % Prevent overflowing lines due to hard-to-break entities
    \sloppy 
    % Setup hyperref package
    \hypersetup{
      breaklinks=true,  % so long urls are correctly broken across lines
      colorlinks=true,
      urlcolor=urlcolor,
      linkcolor=linkcolor,
      citecolor=citecolor,
      }
    % Slightly bigger margins than the latex defaults
    
    \geometry{verbose,tmargin=1in,bmargin=1in,lmargin=1in,rmargin=1in}
    
    

    \begin{document}
    
    
    \maketitle
    
    

    
    \hypertarget{assignment-2-multivariate}{%
\section{Assignment 2 Multivariate}\label{assignment-2-multivariate}}

    \hypertarget{bivariate-normal-distribution-for-polution.csv}{%
\section{\texorpdfstring{1 Bivariate normal distribution for
\texttt{polution.csv}}{1 Bivariate normal distribution for polution.csv}}\label{bivariate-normal-distribution-for-polution.csv}}

    The data file \texttt{pollution.csv} (as on Canvas and SAS Studio)
contains information on air pollution measurements. Using the file
examine the pair of measurements \texttt{X5=Nitrious\ Oxide} and
\texttt{X6=Ozone} for bivariate normality by completing the following:

    \begin{Verbatim}[commandchars=\\\{\}]
{\color{incolor}In [{\color{incolor} }]:} \PY{c}{/* Importing data */}
        \PY{k+kr}{Data }pollutionData;
        	\PY{k}{infile} \PY{l+s}{\PYZdq{}}\PY{l+s}{./data/pollution.csv}\PY{l+s}{\PYZdq{}} \PY{k}{delimiter}=\PY{l+s}{\PYZdq{}}\PY{l+s}{,}\PY{l+s}{\PYZdq{}}; 
        	\PY{k}{input} x1\PYZhy{}x4 NITRIOUSOXIDE OZONE x7\PY{k+kr}{;}
        \PY{k+kr}{	run;}
        \PY{k+kr}{    }
        \PY{k+kr}{proc iml;}
        use pollutionData;
        read all var \PYZob{} NITRIOUSOXIDE OZONE \PYZcb{} \PY{k}{into} pollution;
\end{Verbatim}


    \hypertarget{a-calculate-the-distances-of-these-observations-from-their-means-2-marks}{%
\paragraph{a) Calculate the distances of these observations from their
means (2
marks)}\label{a-calculate-the-distances-of-these-observations-from-their-means-2-marks}}

    \begin{Verbatim}[commandchars=\\\{\}]
{\color{incolor}In [{\color{incolor} }]:} centrePollution=\PY{n+nb}{mean(}pollution); \PY{c}{/*calclate mean vector*/}
        covariancePollution=cov(pollution); \PY{c}{/*calculate cov matrix*/}
        correlationPollution=corr(pollution); \PY{c}{/*calculate correlation matrix*/}
        \PY{c}{/* print centrePollution covariancePollution; */}
        columnVectorPollution = t(pollution\PYZhy{}centrePollution); \PY{c}{/* col vector */}
        distancesPollution = t(columnVectorPollution)*inv(covariancePollution)*columnVectorPollution; \PY{c}{/* calculate distances */}
\end{Verbatim}


    \begin{Verbatim}[commandchars=\\\{\}]
{\color{incolor}In [{\color{incolor} }]:} mahalaPollution=vecdiag(distancesPollution); \PY{c}{/*produce Mahalanobis vector*/}
        print mahalaPollution; \PY{c}{/*produce Mahalanobis vector */}
\end{Verbatim}


    \hypertarget{b-determine-the-proportion-of-the-observations-falling-within-the-estimated-50-probability-contour-of-a-bivariate-normal-distribution-1-mark}{%
\paragraph{\texorpdfstring{b) Determine the proportion of the
observations falling within the estimated
\texttt{50\%\ probability\ contour} of a bivariate normal distribution
(1
mark)}{b) Determine the proportion of the observations falling within the estimated 50\% probability contour of a bivariate normal distribution (1 mark)}}\label{b-determine-the-proportion-of-the-observations-falling-within-the-estimated-50-probability-contour-of-a-bivariate-normal-distribution-1-mark}}

    Bivariate Contour to assess normality

    (Xi - X̄) S-1 (Xi - X̄) ≤ χ2p(ɑ)

    chisquared: χ2p(ɑ) -\textgreater{} χ22(0.5) -\textgreater{} 1.39

    \begin{Verbatim}[commandchars=\\\{\}]
{\color{incolor}In [{\color{incolor}4}]:} pollutionInv = inv(covariancePollution);
        pollutionMean = \PY{n+nb}{mean(}pollution);
        values=(pollution\PYZhy{}pollutionMean)*pollutionInv*t(pollution\PYZhy{}pollutionMean);
        numbers=vecdiag(values);
        inside=numbers\PYZlt{}=\PY{l+m}{1.39};
        outside=numbers\PYZgt{}\PY{l+m}{1.39};
        i\PYZus{}count=\PY{n+nb}{sum(}inside);
        o\PYZus{}count=\PY{n+nb}{sum(}outside);
        \PY{c+cm}{/* print i\PYZus{}count;}
        \PY{c+cm}{print o\PYZus{}count; */}
        proportion = i\PYZus{}count / (i\PYZus{}count + o\PYZus{}count);
        print proportion;
\end{Verbatim}


\begin{Verbatim}[commandchars=\\\{\}]
{\color{outcolor}Out[{\color{outcolor}4}]:} <IPython.core.display.HTML object>
\end{Verbatim}
            
    \hypertarget{c-construct-a-chi-square-plot-of-your-distances-from-part-a-above-2-marks}{%
\paragraph{c) Construct a chi-square plot of your distances from part a)
above (2
marks)}\label{c-construct-a-chi-square-plot-of-your-distances-from-part-a-above-2-marks}}

    \begin{Verbatim}[commandchars=\\\{\}]
{\color{incolor}In [{\color{incolor}5}]:} ranksPollution=\PY{n+nb}{rank(}mahalaPollution); \PY{c}{/*order distances*/}
        pPollution=ncol(pollution); \PY{c}{/*number of columns*/}
        nPollution=nrow(pollution); \PY{c}{/*number of rows*/}
        relativeFrequencyPollution=(ranksPollution\PY{l+m}{\PYZhy{}0.5})/nPollution; \PY{c}{/* Compute the relative frequency */}
        chiSquaredPollution=\PY{n+nb}{cinv(}relativeFrequencyPollution,pPollution); \PY{c}{/* Quantile using Chi\PYZhy{}square distribution with p degrees of freedom */}
        chiplot=mahalaPollution||chiSquaredPollution;
        \PY{k}{create} chiplot \PY{k}{from} chiplot[colname=\PYZob{}\PY{l+s}{\PYZsq{}}\PY{l+s}{MAHDIST}\PY{l+s}{\PYZsq{}} \PY{l+s}{\PYZsq{}}\PY{l+s}{CHISQ}\PY{l+s}{\PYZsq{}}\PYZcb{}]; \PY{c}{/*create a dataset to plot*/}
        append \PY{k}{from} chiplot\PY{k+kr}{;}
        \PY{k+kr}{proc sgplot }data=chiplot;
            scatter y=MAHDIST \PY{k}{x}=CHISQ;
            lineparm \PY{k}{x}=\PY{l+m}{0} y=\PY{l+m}{0} slope=\PY{l+m}{1}\PY{k+kr}{;}
        \PY{k+kr}{run;}
\end{Verbatim}


\begin{Verbatim}[commandchars=\\\{\}]
{\color{outcolor}Out[{\color{outcolor}5}]:} <IPython.core.display.HTML object>
\end{Verbatim}
            
    \begin{Verbatim}[commandchars=\\\{\}]
{\color{incolor}In [{\color{incolor}6}]:} \PY{k+kr}{proc reg }data=chiplot plots=FitPlot(stats=all);
          model MAHDIST = CHISQ;
          ods \PY{k}{select} FitPlot\PY{k+kr}{;}
        \PY{k+kr}{run;}
\end{Verbatim}


\begin{Verbatim}[commandchars=\\\{\}]
{\color{outcolor}Out[{\color{outcolor}6}]:} <IPython.core.display.HTML object>
\end{Verbatim}
            
    \hypertarget{d-given-your-results-in-part-b-and-part-c-are-these-data-approximately-bivariate-normal-explain}{%
\paragraph{d) Given your results in part b) and part c) are these data
approximately bivariate normal?
Explain}\label{d-given-your-results-in-part-b-and-part-c-are-these-data-approximately-bivariate-normal-explain}}

    For bivariate normality we require:

\begin{itemize}
\item
  Multivariate normal distribution (formula)
  \(\displaystyle \mathbf {X} \ \sim \ {\mathcal {N}}_{k}({\mu },\,{\boldsymbol {\Sigma }})\)
  with a single mean
\item
  Representative Sample
\item
  independence of observations
\end{itemize}

Given the Chi Squared Plot results. And with only \textasciitilde{}62\%
of the proportion falling within the estimated
\texttt{50\%\ probability\ contour} of a bivariate normal distribution.

\begin{itemize}
\tightlist
\item
  \textbf{Not bivariate normal}
\end{itemize}

\hypertarget{explain}{%
\subparagraph{Explain}\label{explain}}

\begin{itemize}
\tightlist
\item
  Because observation of qqplot and fit plot show two distinct curves
\item
  Indicates either \textbf{bimodal} or light-tailed results
\item
  \(\displaystyle \mathbf {X} \ \sim \ {\mathcal {N}}_{k}({\boldsymbol {\mu }},\,{\Sigma })\)
  bimodal
\item
  This is even though the results fit within the 95\% CI (95\%)
  according to observation of fit plot
\end{itemize}

    see 3. multivariate normal sect 5 / 6

    \hypertarget{hypothesis-testing---results.csv}{%
\section{\texorpdfstring{2. Hypothesis testing -
\texttt{results.csv}}{2. Hypothesis testing - results.csv}}\label{hypothesis-testing---results.csv}}

    \begin{Verbatim}[commandchars=\\\{\}]
{\color{incolor}In [{\color{incolor}7}]:} \PY{c}{/*read in data*/}
        \PY{k+kr}{Data }temp;
        \PY{k}{infile} \PY{l+s}{\PYZdq{}}\PY{l+s}{./data/results.csv}\PY{l+s}{\PYZdq{}} \PY{k}{delimiter}=\PY{l+s}{\PYZdq{}}\PY{l+s}{,}\PY{l+s}{\PYZdq{}}; 
        \PY{k}{input} x1 x2 x3\PY{k+kr}{;}
        \PY{k+kr}{run;}
        
        \PY{k+kr}{proc IML;}
        use temp;
        read all var \PY{k+kc}{\PYZus{}all\PYZus{}} \PY{k}{into} \PY{k}{X};
\end{Verbatim}


\begin{Verbatim}[commandchars=\\\{\}]
{\color{outcolor}Out[{\color{outcolor}7}]:} <IPython.core.display.HTML object>
\end{Verbatim}
            
    The datafile \texttt{results.csv}, contains three test results assessing
different types of intelligence.

Test the following hypothesis at:

\begin{verbatim}
α=0.02
H0 :μ′=[85 75 55]
\end{verbatim}

\hypertarget{a-conduct-the-hypothesis-test-showing-all-steps-required.}{%
\paragraph{a) Conduct the hypothesis test showing all steps
required.}\label{a-conduct-the-hypothesis-test-showing-all-steps-required.}}

    H0: μ' = μ0' = {[}85 75 55{]} Ha: μ' ≠ μ0' ≠ {[}85 75 55{]}

    \hypertarget{which-kind-of-test-should-we-apply}{%
\paragraph{Which kind of test should we
apply?}\label{which-kind-of-test-should-we-apply}}

    \begin{Verbatim}[commandchars=\\\{\}]
{\color{incolor}In [{\color{incolor}8}]:} n=nrow(\PY{k}{x}); \PY{c}{/* No. of observations */}
        p=ncol(\PY{k}{x}); \PY{c}{/* No. of variables */}
        print n p;
        \PY{c}{/* n=nrow(typesOfIntelligence); /* No. of observations */} */
        \PY{c}{/* p=ncol(typesOfIntelligence); /* No. of variables */} */
\end{Verbatim}


\begin{Verbatim}[commandchars=\\\{\}]
{\color{outcolor}Out[{\color{outcolor}8}]:} <IPython.core.display.HTML object>
\end{Verbatim}
            
    Since n is relatively large, compared to p, we will use the
\texttt{Large\ Sample\ Theory:}

n(X̄n - μ)T Sn-1 (X̄n - μ) ≈ χ2p

    \hypertarget{hypothesis}{%
\paragraph{Hypothesis:}\label{hypothesis}}

H0: μ' = μ0' = {[}85 75 55{]} Ha: μ' ≠ μ0' ≠ {[}85 75 55{]}

    \hypertarget{get-critical-value-from-chi-squared-table}{%
\paragraph{Get critical value from Chi Squared
table}\label{get-critical-value-from-chi-squared-table}}

    χ2p = χ2(80, 0.02) = 108.069

    \hypertarget{do-maths-to-it}{%
\paragraph{Do maths to it}\label{do-maths-to-it}}

    for μ0' = {[}85 75 55{]}

    \begin{Verbatim}[commandchars=\\\{\}]
{\color{incolor}In [{\color{incolor}9}]:} centre=t(\PY{n+nb}{mean(}\PY{k}{X}));
        cov\PYZus{}x=cov(\PY{k}{X});
        cor\PYZus{}x=corr(\PY{k}{X});
        incov=inv(cov\PYZus{}x); \PY{c}{/* Inverse of the covariance matrix */}
        mu0=\PYZob{}\PY{l+m}{85},\PY{l+m}{75},\PY{l+m}{55}\PYZcb{}; \PY{c}{/* Hypothesized values */}
\end{Verbatim}


\begin{Verbatim}[commandchars=\\\{\}]
{\color{outcolor}Out[{\color{outcolor}9}]:} <IPython.core.display.HTML object>
\end{Verbatim}
            
    Large sample: n(X̄n - μ)T Sn-1 (X̄n - μ) ≈ χ2p

    \begin{Verbatim}[commandchars=\\\{\}]
{\color{incolor}In [{\color{incolor}10}]:} tsq=n*t(centre\PYZhy{}mu0)*incov*(centre\PYZhy{}mu0);
         cCriticalValue=\PY{n+nb}{cinv(}\PY{l+m}{0.98},p);
         print tsq cCriticalValue; */
\end{Verbatim}


\begin{Verbatim}[commandchars=\\\{\}]
{\color{outcolor}Out[{\color{outcolor}10}]:} <IPython.core.display.HTML object>
\end{Verbatim}
            
    \hypertarget{if-the-values-given-above-are-the-average-score-for-all-college-students-over-the-last-ten-years-is-there-reason-to-believe-the-group-in-the-datafile-are-scoring-differently-explain.5-marks}{%
\paragraph{If the values given above are the average score for all
college students over the last ten years, is there reason to believe the
group in the datafile are scoring differently? Explain.(5
marks)}\label{if-the-values-given-above-are-the-average-score-for-all-college-students-over-the-last-ten-years-is-there-reason-to-believe-the-group-in-the-datafile-are-scoring-differently-explain.5-marks}}

    \hypertarget{hypothesis}{%
\subparagraph{Hypothesis}\label{hypothesis}}

H0: μ' = μ0' = {[}85 75 55{]} Ha: μ' ≠ μ0' ≠ {[}85 75 55{]}

\hypertarget{assumptions}{%
\subparagraph{Assumptions}\label{assumptions}}

\(\displaystyle \mathbf {X} \ \sim \ {\mathcal {N}}_{k}({\boldsymbol {\mu }},\,{\boldsymbol {\Sigma }})\)

\begin{itemize}
\tightlist
\item
  Representative
\item
  Independence of observerations
\end{itemize}

We \textbf{reject} the null hypothesis because:

Reject if: T2 \textgreater{} χ2p

T2 \textgreater{} χ2p(α=0.02) 51.3 \textgreater{} 9.837 ∴ μ ≠ μ0

\begin{itemize}
\tightlist
\item
  The test score sample mean is sufficiently different during this year
  than the mean (μ' {[}85 75 55{]}) of uni students during previous
  years.
\end{itemize}

    \hypertarget{b-determine-the-lengths-and-directions-for-the-axes-of-the-90-confidence-ellipsoid-for-ux3bc-2-marks} confidence ellipsoid for \texttt{μ} (2
marks)}{b) Determine the lengths and directions for the axes of the 90\% confidence ellipsoid for μ (2 marks)}}\label{b-determine-the-lengths-and-directions-for-the-axes-of-the-90-confidence-ellipsoid-for-ux3bc-2-marks}}

    notes from class /\emph{Compute the lengths of axes of the confidence
ellipsoid}/ lam=eigval(cov\_x); E=eigvec(cov\_x);
length=2\emph{sqrt(lam}ccri\_90percent/n); print length; print E;

    \begin{Verbatim}[commandchars=\\\{\}]
{\color{incolor}In [{\color{incolor}11}]:} centre=t(\PY{n+nb}{mean(}\PY{k}{X}));
         cov\PYZus{}x=cov(\PY{k}{X});
         incov=inv(cov\PYZus{}x); \PY{c}{/* Inverse of the covariance matrix */}
         mu0=\PYZob{}\PY{l+m}{85},\PY{l+m}{75},\PY{l+m}{55}\PYZcb{}; \PY{c}{/* Hypothesized values */}
         cCriticalValue=\PY{n+nb}{cinv(}\PY{l+m}{0.98},p);
         
         lambda=eigval(cov\PYZus{}x); 
         direction\PYZus{}aka\PYZus{}Eigenvector=eigvec(cov\PYZus{}x);
         \PY{k}{length}=\PY{l+m}{2}*\PY{n+nb}{sqrt(}lambda*cCriticalValue/n);
         
         print \PY{k}{length};
         print direction\PYZus{}aka\PYZus{}Eigenvector;
\end{Verbatim}


\begin{Verbatim}[commandchars=\\\{\}]
{\color{outcolor}Out[{\color{outcolor}11}]:} <IPython.core.display.HTML object>
\end{Verbatim}
            
    \hypertarget{c-construct-the-three-possible-scatter-diagrams-from-the-pairs-of-variables.}{%
\paragraph{c) Construct the three possible scatter diagrams from the
pairs of
variables.}\label{c-construct-the-three-possible-scatter-diagrams-from-the-pairs-of-variables.}}

    \begin{Verbatim}[commandchars=\\\{\}]
{\color{incolor}In [{\color{incolor}12}]:} sd=vecdiag(cov\PYZus{}x);
         ubound1=centre+\PY{n+nb}{sqrt(}cCriticalValue)*\PY{n+nb}{sqrt(}sd/n);
         lbound1=centre\PYZhy{}\PY{n+nb}{sqrt(}cCriticalValue)*\PY{n+nb}{sqrt(}sd/n);
         print lbound1 ubound1;
\end{Verbatim}


\begin{Verbatim}[commandchars=\\\{\}]
{\color{outcolor}Out[{\color{outcolor}12}]:} <IPython.core.display.HTML object>
\end{Verbatim}
            
    \begin{Verbatim}[commandchars=\\\{\}]
{\color{incolor}In [{\color{incolor}13}]:} \PY{k+kr}{proc sgscatter }data = temp;
         PLOT X1 * X2\PY{k+kr}{; run;}
\end{Verbatim}


\begin{Verbatim}[commandchars=\\\{\}]
{\color{outcolor}Out[{\color{outcolor}13}]:} <IPython.core.display.HTML object>
\end{Verbatim}
            
    \begin{Verbatim}[commandchars=\\\{\}]
{\color{incolor}In [{\color{incolor}14}]:} \PY{k+kr}{proc sgscatter }data = temp;
         PLOT X1 * X3\PY{k+kr}{; run;}
\end{Verbatim}


\begin{Verbatim}[commandchars=\\\{\}]
{\color{outcolor}Out[{\color{outcolor}14}]:} <IPython.core.display.HTML object>
\end{Verbatim}
            
    \begin{Verbatim}[commandchars=\\\{\}]
{\color{incolor}In [{\color{incolor}15}]:} \PY{k+kr}{proc sgscatter }data = temp;
         PLOT X2 * X3\PY{k+kr}{; run;}
\end{Verbatim}


\begin{Verbatim}[commandchars=\\\{\}]
{\color{outcolor}Out[{\color{outcolor}15}]:} <IPython.core.display.HTML object>
\end{Verbatim}
            
    \hypertarget{do-these-data-appear-to-be-normally-distributed}{%
\paragraph{Do these data appear to be normally
distributed?}\label{do-these-data-appear-to-be-normally-distributed}}

    Nope. This data does not appear to be normally distributed.

    \hypertarget{discuss}{%
\paragraph{Discuss}\label{discuss}}

(3 marks)

    linear gradient = normally distributed

    \hypertarget{question-3}{%
\subsection{Question 3}\label{question-3}}

    \begin{Verbatim}[commandchars=\\\{\}]
{\color{incolor}In [{\color{incolor} }]:} \PY{k+kr}{proc iml;}
        \PY{c}{/* reset print; */}
        dataQ3 = \PYZob{}
        \PY{l+m}{3} \PY{l+m}{4} \PY{l+m}{15} \PYZhy{}\PY{l+m}{6},
        \PY{l+m}{2} \PY{l+m}{4} \PY{l+m}{14} \PYZhy{}\PY{l+m}{7},
        \PY{l+m}{3} \PY{l+m}{4} \PY{l+m}{15} \PYZhy{}\PY{l+m}{5},
        \PY{l+m}{3} \PY{l+m}{3} \PY{l+m}{16} \PYZhy{}\PY{l+m}{6},
        \PY{l+m}{2} \PY{l+m}{5} \PY{l+m}{15} \PYZhy{}\PY{l+m}{7},
        \PY{l+m}{1} \PY{l+m}{4} \PY{l+m}{14} \PYZhy{}\PY{l+m}{4}\PYZcb{};
\end{Verbatim}


    \begin{Verbatim}[commandchars=\\\{\}]
{\color{incolor}In [{\color{incolor} }]:} nQ3=nrow(dataQ3); \PY{c}{/* No. of observations */}
        pQ3=ncol(dataQ3); \PY{c}{/* No. of variables */}
        alphaQ3 = \PY{l+m}{0.05};
        muQ3 = \PY{n+nb}{mean(}dataQ3);
        degreesFreedomQ3 = nQ3 \PYZhy{} \PY{l+m}{1};
        mQ3 = pQ3;
        
        covQ3=cov(dataQ3); \PY{c}{/* covariance */}
        invCovQ3=inv(covQ3); \PY{c}{/* Inverse of the covariance matrix \PYZhy{} Might not need*/}
        sdQ3=vecdiag(CovQ3); \PY{c}{/* get variance from diagonal of the covariance matrix */}
        sdQ3\PYZus{}T=T(vecdiag(CovQ3)); \PY{c}{/* Transposed */}
\end{Verbatim}


    \hypertarget{a-the-independent-95-confidence-intervals-for-each-variable-1.5-marks}{%
\paragraph{a) The independent 95 \% confidence intervals for each
variable (1.5
marks)}\label{a-the-independent-95-confidence-intervals-for-each-variable-1.5-marks}}

    latex problematic formula hidden here: 

    \begin{Verbatim}[commandchars=\\\{\}]
{\color{incolor}In [{\color{incolor} }]:} t\PYZus{}Q3 = \PY{n+nb}{tinv(}(\PY{l+m}{1}\PYZhy{}(alphaQ3/\PY{l+m}{2})), degreesFreedomQ3);
\end{Verbatim}


    \begin{Verbatim}[commandchars=\\\{\}]
{\color{incolor}In [{\color{incolor}20}]:} upperIndependentCIQ3 = muQ3 + t\PYZus{}Q3 * sdQ3\PYZus{}T/\PY{n+nb}{sqrt(}nQ3);
         lowerIndependentCIQ3 = muQ3 \PYZhy{} t\PYZus{}Q3 * sdQ3\PYZus{}T/\PY{n+nb}{sqrt(}nQ3);
         independentConfidenceIntervalsQ3 = (lowerIndependentCIQ3`) || (upperIndependentCIQ3`);
         print independentConfidenceIntervalsQ3[F=\PY{l+m}{8.4} C=\PYZob{}\PY{l+s}{\PYZdq{}}\PY{l+s}{Lower}\PY{l+s}{\PYZdq{}} \PY{l+s}{\PYZdq{}}\PY{l+s}{Upper}\PY{l+s}{\PYZdq{}}\PYZcb{}];
\end{Verbatim}


\begin{Verbatim}[commandchars=\\\{\}]
{\color{outcolor}Out[{\color{outcolor}20}]:} <IPython.core.display.HTML object>
\end{Verbatim}
            
    \hypertarget{b-the-bonferroni-95-confidence-intervals-for-each-variable-1.5-marks}{%
\paragraph{b) The Bonferroni 95 \% confidence intervals for each
variable (1.5
marks)}\label{b-the-bonferroni-95-confidence-intervals-for-each-variable-1.5-marks}}

    latex problematic formula hidden here: 

    latex problematic formula hidden here: 

    \begin{Verbatim}[commandchars=\\\{\}]
{\color{incolor}In [{\color{incolor} }]:} adjustedConfidenceIntervalQ3 = (\PY{l+m}{1} \PYZhy{} (alphaQ3 / (\PY{l+m}{2} * mQ3)));
        \PY{c}{/* print adjustedConfidenceIntervalQ3; */}
\end{Verbatim}


    t(df, alpha) \textless{}- how we write it tinv(CI(aka 1-alpha), df)
\textless{}- back to front

    \begin{Verbatim}[commandchars=\\\{\}]
{\color{incolor}In [{\color{incolor}22}]:} tCriticalValue=\PY{n+nb}{tinv(}adjustedConfidenceIntervalQ3, degreesFreedomQ3);
         \PY{c}{/* print tCriticalValue; */}
\end{Verbatim}


\begin{Verbatim}[commandchars=\\\{\}]
{\color{outcolor}Out[{\color{outcolor}22}]:} <IPython.core.display.HTML object>
\end{Verbatim}
            
    latex problematic formula hidden here: 

    \begin{Verbatim}[commandchars=\\\{\}]
{\color{incolor}In [{\color{incolor}23}]:} sdQ3=vecdiag(CovQ3); \PY{c}{/* this gets the values along the diagonal of the covariance matrix */}
         sdQ3\PYZus{}T=T(vecdiag(CovQ3)); \PY{c}{/* Transposed */}
\end{Verbatim}


\begin{Verbatim}[commandchars=\\\{\}]
{\color{outcolor}Out[{\color{outcolor}23}]:} <IPython.core.display.HTML object>
\end{Verbatim}
            
    \begin{Verbatim}[commandchars=\\\{\}]
{\color{incolor}In [{\color{incolor}24}]:} uboundQ3=muQ3+\PY{n+nb}{sqrt(}tCriticalValue)*\PY{n+nb}{sqrt(}sdQ3\PYZus{}T/nQ3);
         lboundQ3=muQ3\PYZhy{}\PY{n+nb}{sqrt(}tCriticalValue)*\PY{n+nb}{sqrt(}sdQ3\PYZus{}T/nQ3);
         bonferroniConfidenceIntervalsQ3 = (lboundQ3`) || (uboundQ3`);
         print bonferroniConfidenceIntervalsQ3[F=\PY{l+m}{8.4} C=\PYZob{}\PY{l+s}{\PYZdq{}}\PY{l+s}{Lower}\PY{l+s}{\PYZdq{}} \PY{l+s}{\PYZdq{}}\PY{l+s}{Upper}\PY{l+s}{\PYZdq{}}\PYZcb{}];
\end{Verbatim}


\begin{Verbatim}[commandchars=\\\{\}]
{\color{outcolor}Out[{\color{outcolor}24}]:} <IPython.core.display.HTML object>
\end{Verbatim}
            
    \begin{enumerate}
\def\labelenumi{\alph{enumi})}
\setcounter{enumi}{2}
\tightlist
\item
  The simultaneous 95 \% confidence intervals for each variable (1.5
  marks)
\end{enumerate}

    \$\bar \{ \{ X \}\_\{ i \} \}
\pm \sqrt { _{ n-1 }\left( \frac { p(n-1) }{ n-p }  \right) { F }_{ p,n-p }(\alpha )\frac { { S }_{ ii } }{ n }  }
\$

    \begin{Verbatim}[commandchars=\\\{\}]
{\color{incolor}In [{\color{incolor}25}]:} firstPart = \PY{l+m}{1} \PYZhy{} (nQ3 \PYZhy{} pQ3) * alphaQ3;
         tSquared = pQ3 * (nQ3 \PYZhy{} \PY{l+m}{1})/(nQ3 \PYZhy{} pQ3);
         simultaneousCI = \PY{n+nb}{tinv(}firstPart, degreesFreedomQ3);
         
         upperSimultaneousCI = muQ3 + \PY{n+nb}{sqrt(}tSquared * simultaneousCI) * \PY{n+nb}{sqrt(}sdQ3\PYZus{}T / nQ3);
         lowerSimultaneousCI = muQ3 \PYZhy{} \PY{n+nb}{sqrt(}tSquared * simultaneousCI) * \PY{n+nb}{sqrt(}sdQ3\PYZus{}T / nQ3);
         
         simultaneousConfidenceIntervalsQ3 = (lowerSimultaneousCI`) || (upperSimultaneousCI`);
         print simultaneousConfidenceIntervalsQ3[F=\PY{l+m}{8.4} C=\PYZob{}\PY{l+s}{\PYZdq{}}\PY{l+s}{Lower}\PY{l+s}{\PYZdq{}} \PY{l+s}{\PYZdq{}}\PY{l+s}{Upper}\PY{l+s}{\PYZdq{}}\PYZcb{}];
\end{Verbatim}


\begin{Verbatim}[commandchars=\\\{\}]
{\color{outcolor}Out[{\color{outcolor}25}]:} <IPython.core.display.HTML object>
\end{Verbatim}
            
    \begin{enumerate}
\def\labelenumi{\alph{enumi})}
\setcounter{enumi}{3}
\tightlist
\item
  The 95 \% confidence interval for the difference between μ2 and μ4.
  Are these means different? (1.5 marks)
\end{enumerate}

    \begin{Verbatim}[commandchars=\\\{\}]
{\color{incolor}In [{\color{incolor}26}]:} muQ3X2 = \PY{n+nb}{mean(}dataQ3[\PY{l+m}{2}]);
         muQ3X4 = \PY{n+nb}{mean(}dataQ3[\PY{l+m}{4}]);
         diffXbar = (muQ3X2 \PYZhy{} muQ3X4);
         secondPartDiffEquation = \PY{n+nb}{sqrt(}tSquared * sdQ3\PYZus{}T) * \PY{n+nb}{sqrt(}(covQ3[\PY{l+m}{2},\PY{l+m}{2}] \PYZhy{} \PY{l+m}{2} * covQ3[\PY{l+m}{2},\PY{l+m}{4}] + covQ3[\PY{l+m}{4},\PY{l+m}{4}]) / nQ3);
         
         upperDiffCIMu2Mu4Q3 = diffXbar + secondPartDiffEquation;
         lowerDiffCIMu2Mu4Q3 = diffXbar \PYZhy{} secondPartDiffEquation;
         
         
         diffCIMu2Mu4Q3 = (lowerDiffCIMu2Mu4Q3`) || (upperDiffCIMu2Mu4Q3`);
         print diffCIMu2Mu4Q3[F=\PY{l+m}{8.4} C=\PYZob{}\PY{l+s}{\PYZdq{}}\PY{l+s}{Lower}\PY{l+s}{\PYZdq{}} \PY{l+s}{\PYZdq{}}\PY{l+s}{Upper}\PY{l+s}{\PYZdq{}}\PYZcb{}];
         
         
         \PY{c}{/* The CI for μ2 \PYZhy{} μ4 does not include zero, therefore the muQ3s are different */}
\end{Verbatim}


\begin{Verbatim}[commandchars=\\\{\}]
{\color{outcolor}Out[{\color{outcolor}26}]:} <IPython.core.display.HTML object>
\end{Verbatim}
            
    \begin{enumerate}
\def\labelenumi{\alph{enumi})}
\setcounter{enumi}{4}
\tightlist
\item
  Discuss the results from part a) to part d) above and explain any
  differences in the observed estimates. (2 marks)
\end{enumerate}

    \begin{Verbatim}[commandchars=\\\{\}]
{\color{incolor}In [{\color{incolor}27}]:} muQ3\PYZus{}t = t(muQ3);
         print muQ3\PYZus{}t independentConfidenceIntervalsQ3[F=\PY{l+m}{8.4} C=\PYZob{}\PY{l+s}{\PYZdq{}}\PY{l+s}{Lower}\PY{l+s}{\PYZdq{}} \PY{l+s}{\PYZdq{}}\PY{l+s}{Upper}\PY{l+s}{\PYZdq{}}\PYZcb{}];
\end{Verbatim}


\begin{Verbatim}[commandchars=\\\{\}]
{\color{outcolor}Out[{\color{outcolor}27}]:} <IPython.core.display.HTML object>
\end{Verbatim}
            
    \hypertarget{pcas-with---track.csv}{%
\subsection{\texorpdfstring{4 PCA's with -
\texttt{track.csv}}{4 PCA's with - track.csv}}\label{pcas-with---track.csv}}

see other file for full SAS code for question 4 due to small bug with
jupyter with SAS. - The data file \texttt{track.csv} contains
information on female national track records. Using the file completing
the following:

    \%web\_drop\_table(R); FILENAME REFFILE
`/folders/myshortcuts/ass2--multivariate/data/track.csv'; PROC IMPORT
DATAFILE=REFFILE DBMS=CSV OUT=R; /* this is where we get the output with
the PCA's */ GETNAMES=YES; RUN; PROC CONTENTS DATA=R; RUN;
\%web\_open\_table(R); 

    \hypertarget{a-pt1-obtain-the-sample-correlation-matrix-r}{%
\paragraph{a) pt1 Obtain the sample correlation matrix
R}\label{a-pt1-obtain-the-sample-correlation-matrix-r}}

     proc IML; proc princomp data=R OUT=prinR; title3 `4 a) Obtain the
sample correlation matrix R, and determine the eigenvalue/eigenvector
pairs.'; run;

proc IML; /* fix printing error \emph{/ }';*"; run; proc iml; title3 `4
b) pt1) State the first two principal components for the standardized
variables'; print `PCA1 Y1 = 0.377766X1 + 0.383210X2 + 0.368036X3 +
0.394781X4 + 0.389261X5 + 0.376094X6 + 0.355203X7'; print `PCA2 Y2 =
-0.407176X1 - 0.413629X2 - 0.459353X3 + 0.161246X4 + 0.309088X5 +
0.423190X6 + 0.389215X7'; run; 

    \begin{longtable}[]{@{}llllllll@{}}
\toprule
\begin{minipage}[b]{0.10\columnwidth}\raggedright
\_100m\_\_s\_\strut
\end{minipage} & \begin{minipage}[b]{0.10\columnwidth}\raggedright
\_200m\_\_s\_\strut
\end{minipage} & \begin{minipage}[b]{0.10\columnwidth}\raggedright
\_400m\_\_s\_\strut
\end{minipage} & \begin{minipage}[b]{0.10\columnwidth}\raggedright
\_800m\_\_min\_\strut
\end{minipage} & \begin{minipage}[b]{0.10\columnwidth}\raggedright
\_1500m\_\_min\_\strut
\end{minipage} & \begin{minipage}[b]{0.10\columnwidth}\raggedright
\_3000m\_\_min\_\strut
\end{minipage} & \begin{minipage}[b]{0.10\columnwidth}\raggedright
Marathon\_\emph{min}\strut
\end{minipage} & \begin{minipage}[b]{0.10\columnwidth}\raggedright
\strut
\end{minipage}\tabularnewline
\midrule
\endhead
\begin{minipage}[t]{0.10\columnwidth}\raggedright
\_100m\_\_s\_\strut
\end{minipage} & \begin{minipage}[t]{0.10\columnwidth}\raggedright
1.0000\strut
\end{minipage} & \begin{minipage}[t]{0.10\columnwidth}\raggedright
0.9411\strut
\end{minipage} & \begin{minipage}[t]{0.10\columnwidth}\raggedright
0.8708\strut
\end{minipage} & \begin{minipage}[t]{0.10\columnwidth}\raggedright
0.8092\strut
\end{minipage} & \begin{minipage}[t]{0.10\columnwidth}\raggedright
0.7816\strut
\end{minipage} & \begin{minipage}[t]{0.10\columnwidth}\raggedright
0.7279\strut
\end{minipage} & \begin{minipage}[t]{0.10\columnwidth}\raggedright
0.6690\strut
\end{minipage}\tabularnewline
\begin{minipage}[t]{0.10\columnwidth}\raggedright
\_200m\_\_s\_\strut
\end{minipage} & \begin{minipage}[t]{0.10\columnwidth}\raggedright
0.9411\strut
\end{minipage} & \begin{minipage}[t]{0.10\columnwidth}\raggedright
1.0000\strut
\end{minipage} & \begin{minipage}[t]{0.10\columnwidth}\raggedright
0.9088\strut
\end{minipage} & \begin{minipage}[t]{0.10\columnwidth}\raggedright
0.8198\strut
\end{minipage} & \begin{minipage}[t]{0.10\columnwidth}\raggedright
0.8013\strut
\end{minipage} & \begin{minipage}[t]{0.10\columnwidth}\raggedright
0.7319\strut
\end{minipage} & \begin{minipage}[t]{0.10\columnwidth}\raggedright
0.6800\strut
\end{minipage}\tabularnewline
\begin{minipage}[t]{0.10\columnwidth}\raggedright
\_400m\_\_s\_\strut
\end{minipage} & \begin{minipage}[t]{0.10\columnwidth}\raggedright
0.8708\strut
\end{minipage} & \begin{minipage}[t]{0.10\columnwidth}\raggedright
0.9088\strut
\end{minipage} & \begin{minipage}[t]{0.10\columnwidth}\raggedright
1.0000\strut
\end{minipage} & \begin{minipage}[t]{0.10\columnwidth}\raggedright
0.8058\strut
\end{minipage} & \begin{minipage}[t]{0.10\columnwidth}\raggedright
0.7198\strut
\end{minipage} & \begin{minipage}[t]{0.10\columnwidth}\raggedright
0.6738\strut
\end{minipage} & \begin{minipage}[t]{0.10\columnwidth}\raggedright
0.6769\strut
\end{minipage}\tabularnewline
\begin{minipage}[t]{0.10\columnwidth}\raggedright
\_800m\_\_min\_\strut
\end{minipage} & \begin{minipage}[t]{0.10\columnwidth}\raggedright
0.8092\strut
\end{minipage} & \begin{minipage}[t]{0.10\columnwidth}\raggedright
0.8198\strut
\end{minipage} & \begin{minipage}[t]{0.10\columnwidth}\raggedright
0.8058\strut
\end{minipage} & \begin{minipage}[t]{0.10\columnwidth}\raggedright
1.0000\strut
\end{minipage} & \begin{minipage}[t]{0.10\columnwidth}\raggedright
0.9051\strut
\end{minipage} & \begin{minipage}[t]{0.10\columnwidth}\raggedright
0.8666\strut
\end{minipage} & \begin{minipage}[t]{0.10\columnwidth}\raggedright
0.8540\strut
\end{minipage}\tabularnewline
\begin{minipage}[t]{0.10\columnwidth}\raggedright
\_1500m\_\_min\_\strut
\end{minipage} & \begin{minipage}[t]{0.10\columnwidth}\raggedright
0.7816\strut
\end{minipage} & \begin{minipage}[t]{0.10\columnwidth}\raggedright
0.8013\strut
\end{minipage} & \begin{minipage}[t]{0.10\columnwidth}\raggedright
0.7198\strut
\end{minipage} & \begin{minipage}[t]{0.10\columnwidth}\raggedright
0.9051\strut
\end{minipage} & \begin{minipage}[t]{0.10\columnwidth}\raggedright
1.0000\strut
\end{minipage} & \begin{minipage}[t]{0.10\columnwidth}\raggedright
0.9734\strut
\end{minipage} & \begin{minipage}[t]{0.10\columnwidth}\raggedright
0.7906\strut
\end{minipage}\tabularnewline
\begin{minipage}[t]{0.10\columnwidth}\raggedright
\_3000m\_\_min\_\strut
\end{minipage} & \begin{minipage}[t]{0.10\columnwidth}\raggedright
0.7279\strut
\end{minipage} & \begin{minipage}[t]{0.10\columnwidth}\raggedright
0.7319\strut
\end{minipage} & \begin{minipage}[t]{0.10\columnwidth}\raggedright
0.6738\strut
\end{minipage} & \begin{minipage}[t]{0.10\columnwidth}\raggedright
0.8666\strut
\end{minipage} & \begin{minipage}[t]{0.10\columnwidth}\raggedright
0.9734\strut
\end{minipage} & \begin{minipage}[t]{0.10\columnwidth}\raggedright
1.0000\strut
\end{minipage} & \begin{minipage}[t]{0.10\columnwidth}\raggedright
0.7987\strut
\end{minipage}\tabularnewline
\begin{minipage}[t]{0.10\columnwidth}\raggedright
Marathon\_\emph{min}\strut
\end{minipage} & \begin{minipage}[t]{0.10\columnwidth}\raggedright
0.6690\strut
\end{minipage} & \begin{minipage}[t]{0.10\columnwidth}\raggedright
0.6800\strut
\end{minipage} & \begin{minipage}[t]{0.10\columnwidth}\raggedright
0.6769\strut
\end{minipage} & \begin{minipage}[t]{0.10\columnwidth}\raggedright
0.8540\strut
\end{minipage} & \begin{minipage}[t]{0.10\columnwidth}\raggedright
0.7906\strut
\end{minipage} & \begin{minipage}[t]{0.10\columnwidth}\raggedright
0.7987\strut
\end{minipage} & \begin{minipage}[t]{0.10\columnwidth}\raggedright
1.0000\strut
\end{minipage}\tabularnewline
\bottomrule
\end{longtable}

    \hypertarget{a-pt2-determine-the-eigenvalueeigenvector-pairs.}{%
\paragraph{a) pt2 determine the eigenvalue/eigenvector
pairs.}\label{a-pt2-determine-the-eigenvalueeigenvector-pairs.}}

    \begin{longtable}[]{@{}llllllllll@{}}
\toprule
\begin{minipage}[b]{0.07\columnwidth}\raggedright
\strut
\end{minipage} & \begin{minipage}[b]{0.07\columnwidth}\raggedright
\strut
\end{minipage} & \begin{minipage}[b]{0.07\columnwidth}\raggedright
Eigenvalue\strut
\end{minipage} & \begin{minipage}[b]{0.07\columnwidth}\raggedright
Eigen - Prin1\strut
\end{minipage} & \begin{minipage}[b]{0.07\columnwidth}\raggedright
Eigen - Prin2\strut
\end{minipage} & \begin{minipage}[b]{0.07\columnwidth}\raggedright
Eigen - Prin3\strut
\end{minipage} & \begin{minipage}[b]{0.07\columnwidth}\raggedright
Eigen - Prin4\strut
\end{minipage} & \begin{minipage}[b]{0.07\columnwidth}\raggedright
Eigen - Prin5\strut
\end{minipage} & \begin{minipage}[b]{0.07\columnwidth}\raggedright
Eigen - Prin6\strut
\end{minipage} & \begin{minipage}[b]{0.07\columnwidth}\raggedright
Eigen - Prin7\strut
\end{minipage}\tabularnewline
\midrule
\endhead
\begin{minipage}[t]{0.07\columnwidth}\raggedright
x1\strut
\end{minipage} & \begin{minipage}[t]{0.07\columnwidth}\raggedright
\_100m\_\_s\_\strut
\end{minipage} & \begin{minipage}[t]{0.07\columnwidth}\raggedright
\textbf{5.80762446}\strut
\end{minipage} & \begin{minipage}[t]{0.07\columnwidth}\raggedright
0.377766\strut
\end{minipage} & \begin{minipage}[t]{0.07\columnwidth}\raggedright
-.407176\strut
\end{minipage} & \begin{minipage}[t]{0.07\columnwidth}\raggedright
-.140580\strut
\end{minipage} & \begin{minipage}[t]{0.07\columnwidth}\raggedright
0.587063\strut
\end{minipage} & \begin{minipage}[t]{0.07\columnwidth}\raggedright
-.167069\strut
\end{minipage} & \begin{minipage}[t]{0.07\columnwidth}\raggedright
0.539697\strut
\end{minipage} & \begin{minipage}[t]{0.07\columnwidth}\raggedright
-.088939\strut
\end{minipage}\tabularnewline
\begin{minipage}[t]{0.07\columnwidth}\raggedright
x2\strut
\end{minipage} & \begin{minipage}[t]{0.07\columnwidth}\raggedright
\_200m\_\_s\_\strut
\end{minipage} & \begin{minipage}[t]{0.07\columnwidth}\raggedright
\textbf{0.62869342}\strut
\end{minipage} & \begin{minipage}[t]{0.07\columnwidth}\raggedright
0.383210\strut
\end{minipage} & \begin{minipage}[t]{0.07\columnwidth}\raggedright
-.413629\strut
\end{minipage} & \begin{minipage}[t]{0.07\columnwidth}\raggedright
-.100783\strut
\end{minipage} & \begin{minipage}[t]{0.07\columnwidth}\raggedright
0.194075\strut
\end{minipage} & \begin{minipage}[t]{0.07\columnwidth}\raggedright
0.093500\strut
\end{minipage} & \begin{minipage}[t]{0.07\columnwidth}\raggedright
-.744931\strut
\end{minipage} & \begin{minipage}[t]{0.07\columnwidth}\raggedright
0.265657\strut
\end{minipage}\tabularnewline
\begin{minipage}[t]{0.07\columnwidth}\raggedright
x3\strut
\end{minipage} & \begin{minipage}[t]{0.07\columnwidth}\raggedright
\_400m\_\_s\_\strut
\end{minipage} & \begin{minipage}[t]{0.07\columnwidth}\raggedright
\textbf{0.27933457}\strut
\end{minipage} & \begin{minipage}[t]{0.07\columnwidth}\raggedright
0.368036\strut
\end{minipage} & \begin{minipage}[t]{0.07\columnwidth}\raggedright
-.459353\strut
\end{minipage} & \begin{minipage}[t]{0.07\columnwidth}\raggedright
0.237026\strut
\end{minipage} & \begin{minipage}[t]{0.07\columnwidth}\raggedright
-.645431\strut
\end{minipage} & \begin{minipage}[t]{0.07\columnwidth}\raggedright
0.327273\strut
\end{minipage} & \begin{minipage}[t]{0.07\columnwidth}\raggedright
0.240094\strut
\end{minipage} & \begin{minipage}[t]{0.07\columnwidth}\raggedright
-.126604\strut
\end{minipage}\tabularnewline
\begin{minipage}[t]{0.07\columnwidth}\raggedright
x4\strut
\end{minipage} & \begin{minipage}[t]{0.07\columnwidth}\raggedright
\_800m\_\_min\_\strut
\end{minipage} & \begin{minipage}[t]{0.07\columnwidth}\raggedright
\textbf{0.12455472}\strut
\end{minipage} & \begin{minipage}[t]{0.07\columnwidth}\raggedright
0.394781\strut
\end{minipage} & \begin{minipage}[t]{0.07\columnwidth}\raggedright
0.161246\strut
\end{minipage} & \begin{minipage}[t]{0.07\columnwidth}\raggedright
0.147542\strut
\end{minipage} & \begin{minipage}[t]{0.07\columnwidth}\raggedright
-.295208\strut
\end{minipage} & \begin{minipage}[t]{0.07\columnwidth}\raggedright
-.819055\strut
\end{minipage} & \begin{minipage}[t]{0.07\columnwidth}\raggedright
-.016507\strut
\end{minipage} & \begin{minipage}[t]{0.07\columnwidth}\raggedright
0.195213\strut
\end{minipage}\tabularnewline
\begin{minipage}[t]{0.07\columnwidth}\raggedright
x5\strut
\end{minipage} & \begin{minipage}[t]{0.07\columnwidth}\raggedright
\_1500m\_\_min\_\strut
\end{minipage} & \begin{minipage}[t]{0.07\columnwidth}\raggedright
\textbf{0.09097174}\strut
\end{minipage} & \begin{minipage}[t]{0.07\columnwidth}\raggedright
0.389261\strut
\end{minipage} & \begin{minipage}[t]{0.07\columnwidth}\raggedright
0.309088\strut
\end{minipage} & \begin{minipage}[t]{0.07\columnwidth}\raggedright
-.421986\strut
\end{minipage} & \begin{minipage}[t]{0.07\columnwidth}\raggedright
-.066690\strut
\end{minipage} & \begin{minipage}[t]{0.07\columnwidth}\raggedright
0.026131\strut
\end{minipage} & \begin{minipage}[t]{0.07\columnwidth}\raggedright
-.188988\strut
\end{minipage} & \begin{minipage}[t]{0.07\columnwidth}\raggedright
-.730768\strut
\end{minipage}\tabularnewline
\begin{minipage}[t]{0.07\columnwidth}\raggedright
x6\strut
\end{minipage} & \begin{minipage}[t]{0.07\columnwidth}\raggedright
\_3000m\_\_min\_\strut
\end{minipage} & \begin{minipage}[t]{0.07\columnwidth}\raggedright
\textbf{0.05451882}\strut
\end{minipage} & \begin{minipage}[t]{0.07\columnwidth}\raggedright
0.376094\strut
\end{minipage} & \begin{minipage}[t]{0.07\columnwidth}\raggedright
0.423190\strut
\end{minipage} & \begin{minipage}[t]{0.07\columnwidth}\raggedright
-.406063\strut
\end{minipage} & \begin{minipage}[t]{0.07\columnwidth}\raggedright
-.080157\strut
\end{minipage} & \begin{minipage}[t]{0.07\columnwidth}\raggedright
0.351698\strut
\end{minipage} & \begin{minipage}[t]{0.07\columnwidth}\raggedright
0.240500\strut
\end{minipage} & \begin{minipage}[t]{0.07\columnwidth}\raggedright
0.571506\strut
\end{minipage}\tabularnewline
\begin{minipage}[t]{0.07\columnwidth}\raggedright
x7\strut
\end{minipage} & \begin{minipage}[t]{0.07\columnwidth}\raggedright
Marathon\_\emph{min}\strut
\end{minipage} & \begin{minipage}[t]{0.07\columnwidth}\raggedright
\textbf{0.01430226}\strut
\end{minipage} & \begin{minipage}[t]{0.07\columnwidth}\raggedright
0.355203\strut
\end{minipage} & \begin{minipage}[t]{0.07\columnwidth}\raggedright
0.389215\strut
\end{minipage} & \begin{minipage}[t]{0.07\columnwidth}\raggedright
0.741061\strut
\end{minipage} & \begin{minipage}[t]{0.07\columnwidth}\raggedright
0.321076\strut
\end{minipage} & \begin{minipage}[t]{0.07\columnwidth}\raggedright
0.247008\strut
\end{minipage} & \begin{minipage}[t]{0.07\columnwidth}\raggedright
-.048270\strut
\end{minipage} & \begin{minipage}[t]{0.07\columnwidth}\raggedright
-.082084\strut
\end{minipage}\tabularnewline
\bottomrule
\end{longtable}

    \hypertarget{b-state-the-first-two-principal-components-for-the-standardized-variables}{%
\paragraph{b) State the first two principal components for the
standardized
variables}\label{b-state-the-first-two-principal-components-for-the-standardized-variables}}

    \begin{itemize}
\tightlist
\item
  PCA1 Y1 = 0.377766X1 + 0.383210X2 + 0.368036X3 + 0.394781X4 +
  0.389261X5 + 0.376094X6 + 0.355203X7
\item
  PCA2 Y2 = -0.407176X1 - 0.413629X2 - 0.459353X3 + 0.161246X4 +
  0.309088X5 + 0.423190X6 + 0.389215X7
\end{itemize}

     data Cumulative; length CumulativePercent \$ 2; input X \$1-15
Eigenvalue \$1-15 CumulativePercentCol; CumulativePercent =
substr(X,1,2); datalines; X Eigenvalue Cumulative X1 5.80762446 82.97 X2
0.62869342 91.95 X3 0.27933457 95.94 X4 0.12455472 97.72 X5 0.09097174
99.02 X6 0.05451882 99.80 X7 0.01430226 100.00 ; proc print
data=Cumulative; title3 `4 b) pt2. Calculate the cumulative percentages
of the total (standardized) sample variance explained.'; run; 

    \hypertarget{b-pt2.-calculate-the-cumulative-percentages-of-the-total-standardized-sample-variance-explained.-2-marks}{%
\paragraph{b) pt2. calculate the cumulative percentages of the total
(standardized) sample variance explained. (2
marks)}\label{b-pt2.-calculate-the-cumulative-percentages-of-the-total-standardized-sample-variance-explained.-2-marks}}

    \begin{longtable}[]{@{}lll@{}}
\toprule
PCA & Eigenvalue & CumulativePercent\tabularnewline
\midrule
\endhead
X1 & 5.80762446 & 82.97\tabularnewline
X2 & 0.62869342 & 91.95\tabularnewline
X3 & 0.27933457 & 95.94\tabularnewline
X4 & 0.12455472 & 97.72\tabularnewline
X5 & 0.09097174 & 99.02\tabularnewline
X6 & 0.05451882 & 99.80\tabularnewline
X7 & 0.01430226 & 100.00\tabularnewline
\bottomrule
\end{longtable}

    \hypertarget{c-prepare-a-table-showing-the-correlation-of-the-standardized-variables-with-the-first-two-components.}{%
\paragraph{c) Prepare a table showing the correlation of the
standardized variables with the first two
components.}\label{c-prepare-a-table-showing-the-correlation-of-the-standardized-variables-with-the-first-two-components.}}

     proc princomp data=R n=2 outstat=standardVariablesCorr noprint; run;
proc print data = standardVariablesCorr; title3 ``c) Prepare a table
showing the correlation of the standardized variables with the first two
components.''; where \emph{TYPE} = `SCORE'; run;

/* proc print data=R; \emph{/ /} run; \emph{/ /} proc print data=prinR;
\emph{/ /} run; */

proc print data=prinR; var Country Prin1 Prin2; /* filter to only
display country and first Principle Component run; 

    \begin{longtable}[]{@{}llllllll@{}}
\toprule
\begin{minipage}[b]{0.10\columnwidth}\raggedright
PCA\strut
\end{minipage} & \begin{minipage}[b]{0.10\columnwidth}\raggedright
\_100m\_\_s\_\strut
\end{minipage} & \begin{minipage}[b]{0.10\columnwidth}\raggedright
\_200m\_\_s\_\strut
\end{minipage} & \begin{minipage}[b]{0.10\columnwidth}\raggedright
\_400m\_\_s\_\strut
\end{minipage} & \begin{minipage}[b]{0.10\columnwidth}\raggedright
\_800m\_\_min\_\strut
\end{minipage} & \begin{minipage}[b]{0.10\columnwidth}\raggedright
\_1500m\_\_min\_\strut
\end{minipage} & \begin{minipage}[b]{0.10\columnwidth}\raggedright
\_3000m\_\_min\_\strut
\end{minipage} & \begin{minipage}[b]{0.10\columnwidth}\raggedright
Marathon\_\emph{min}\strut
\end{minipage}\tabularnewline
\midrule
\endhead
\begin{minipage}[t]{0.10\columnwidth}\raggedright
Prin1\strut
\end{minipage} & \begin{minipage}[t]{0.10\columnwidth}\raggedright
0.37777\strut
\end{minipage} & \begin{minipage}[t]{0.10\columnwidth}\raggedright
0.38321\strut
\end{minipage} & \begin{minipage}[t]{0.10\columnwidth}\raggedright
0.36804\strut
\end{minipage} & \begin{minipage}[t]{0.10\columnwidth}\raggedright
0.39478\strut
\end{minipage} & \begin{minipage}[t]{0.10\columnwidth}\raggedright
0.38926\strut
\end{minipage} & \begin{minipage}[t]{0.10\columnwidth}\raggedright
0.37609\strut
\end{minipage} & \begin{minipage}[t]{0.10\columnwidth}\raggedright
0.35520\strut
\end{minipage}\tabularnewline
\begin{minipage}[t]{0.10\columnwidth}\raggedright
Prin2\strut
\end{minipage} & \begin{minipage}[t]{0.10\columnwidth}\raggedright
-0.40718\strut
\end{minipage} & \begin{minipage}[t]{0.10\columnwidth}\raggedright
-0.41363\strut
\end{minipage} & \begin{minipage}[t]{0.10\columnwidth}\raggedright
-0.45935\strut
\end{minipage} & \begin{minipage}[t]{0.10\columnwidth}\raggedright
0.16125\strut
\end{minipage} & \begin{minipage}[t]{0.10\columnwidth}\raggedright
0.30909\strut
\end{minipage} & \begin{minipage}[t]{0.10\columnwidth}\raggedright
0.42319\strut
\end{minipage} & \begin{minipage}[t]{0.10\columnwidth}\raggedright
0.38922\strut
\end{minipage}\tabularnewline
\bottomrule
\end{longtable}

    \hypertarget{d-interpret-the-two-principal-components-from-part-b.-2-marks-3-marks}{%
\paragraph{d) Interpret the two principal components from Part b). (2
marks) (3
marks)}\label{d-interpret-the-two-principal-components-from-part-b.-2-marks-3-marks}}

    \begin{itemize}
\tightlist
\item
  The first principle component accounts for \textasciitilde{}82.97\% of
  female national track records
\item
  Second principle component accounts for an additonal
  \textasciitilde{}8.98\% of female national track records
\item
  The dataset can be reduced to 2 principle components to explain
  \textasciitilde{}91.95\% of female national track records.
\end{itemize}

    \hypertarget{e-rank-the-nations-based-on-their-score-on-the-first-principal-component.}{%
\paragraph{e) Rank the nations based on their score on the first
principal
component.}\label{e-rank-the-nations-based-on-their-score-on-the-first-principal-component.}}

     proc sort data=prinR; /* rank by principle component \#1 \emph{/ by
Prin1; run; proc print data=prinR; var Country Prin1; /} filter to only
display country and first Principle Component */ title `e) Rank the
nations based on their score on the first principal component'; run; 

    \begin{longtable}[]{@{}lll@{}}
\toprule
Obs & Country & Prin1\tabularnewline
\midrule
\endhead
1 & ARG & 0.39324\tabularnewline
2 & AUS & -1.93164\tabularnewline
3 & AUT & -1.26252\tabularnewline
4 & BEL & -1.29173\tabularnewline
5 & BER & 1.39611\tabularnewline
6 & BRA & -1.00678\tabularnewline
7 & CAN & -1.73434\tabularnewline
8 & CHI & 0.81184\tabularnewline
9 & CHN & -2.98947\tabularnewline
10 & COL & 0.00193\tabularnewline
11 & COK & 7.90623\tabularnewline
12 & CRC & 2.16681\tabularnewline
13 & CZE & -2.40603\tabularnewline
14 & DEN & -0.08250\tabularnewline
15 & DOM & 2.19241\tabularnewline
16 & FIN & -1.26673\tabularnewline
17 & FRA & -2.51835\tabularnewline
18 & GER & -3.04752\tabularnewline
19 & GBR & -2.44271\tabularnewline
20 & GRE & -1.19780\tabularnewline
21 & GUA & 3.29412\tabularnewline
22 & HUN & -0.78825\tabularnewline
23 & INA & 1.74194\tabularnewline
24 & IND & -0.35426\tabularnewline
25 & IRL & -1.03591\tabularnewline
26 & ISR & 0.57416\tabularnewline
27 & ITA & -1.54745\tabularnewline
28 & JPN & -0.48166\tabularnewline
29 & KEN & -0.91774\tabularnewline
30 & KOR & 0.83079\tabularnewline
31 & KOR & 1.45535\tabularnewline
32 & LUX & 1.72147\tabularnewline
33 & MAS & 1.49521\tabularnewline
34 & MRI & 1.74973\tabularnewline
35 & MEX & -0.99577\tabularnewline
36 & MYA & 0.81598\tabularnewline
37 & NED & -1.54476\tabularnewline
38 & NZL & -0.75524\tabularnewline
39 & NOR & -0.55300\tabularnewline
40 & PNG & 5.25745\tabularnewline
41 & PHI & 1.76353\tabularnewline
42 & POL & -2.27377\tabularnewline
43 & POR & -1.17525\tabularnewline
44 & ROM & -2.12301\tabularnewline
45 & RUS & -3.04295\tabularnewline
46 & SAM & 8.21342\tabularnewline
47 & SIN & 3.09392\tabularnewline
48 & ESP & -1.88946\tabularnewline
49 & SWE & -0.83915\tabularnewline
50 & SUI & -1.11355\tabularnewline
51 & TPE & 0.65909\tabularnewline
52 & THA & 1.22381\tabularnewline
53 & TUR & -0.85013\tabularnewline
54 & USA & -3.29915\tabularnewline
\bottomrule
\end{longtable}

    \hypertarget{discuss-whether-this-meets-your-expectations.-2-marks}{%
\paragraph{Discuss whether this meets your expectations. (2
marks)}\label{discuss-whether-this-meets-your-expectations.-2-marks}}

    Doesn't meet my expectations because current world records for track
records are:

\begin{itemize}
\tightlist
\item
  Women's records are from USA, France, Soviet, Romania, Kenya, Ethiopia
\item
  Men all from Africa (Jamaica, South Africa, Kenya, Ethiopia)
\end{itemize}

Possibly because date of the records isn't supplied. Further
investigation could compare current world records accross all countries

    \hypertarget{ranked-by-principle-component-1-accross-all-track-records}{%
\subparagraph{Ranked by Principle Component 1 (accross all track
records)}\label{ranked-by-principle-component-1-accross-all-track-records}}

\begin{longtable}[]{@{}lll@{}}
\toprule
Obs & Country & Prin1\tabularnewline
\midrule
\endhead
1 & ARG & 0.39324\tabularnewline
2 & AUS & -1.93164\tabularnewline
3 & AUT & -1.26252\tabularnewline
4 & BEL & -1.29173\tabularnewline
5 & BER & 1.39611\tabularnewline
6 & BRA & -1.00678\tabularnewline
7 & CAN & -1.73434\tabularnewline
8 & CHI & 0.81184\tabularnewline
9 & CHN & -2.98947\tabularnewline
10 & COL & 0.00193\tabularnewline
\bottomrule
\end{longtable}

\hypertarget{mens-records}{%
\subparagraph{\texorpdfstring{\href{https://en.wikipedia.org/wiki/List_of_Olympic_records_in_athletics\#Men's_records}{Men's
records}}{Men's records}}\label{mens-records}}

\begin{longtable}[]{@{}llll@{}}
\toprule
\endhead
100 metres & 0:9.63 & Usain Bolt & Jamaica\tabularnewline
200 metres & 0:19.30 & Usain Bolt & Jamaica\tabularnewline
400 metres & 0:43.03 & Wayde van Niekerk & South Africa\tabularnewline
800 metres & 1:40.91 & David Rudisha & Kenya\tabularnewline
1500 metres & 3:32.07 & Noah Ngeny & Kenya\tabularnewline
5000 metres & 12:57.82 & Kenenisa Bekele & Ethiopia\tabularnewline
10000 metre & 27:01.17 & Kenenisa Bekele & Ethiopia\tabularnewline
Marathon & 2:06:32 & Samuel Wanjiru & Kenya\tabularnewline
\bottomrule
\end{longtable}

\hypertarget{womens-records}{%
\subparagraph{\texorpdfstring{\href{https://en.wikipedia.org/wiki/List_of_Olympic_records_in_athletics\#Women's_records}{Women's
records}}{Women's records}}\label{womens-records}}

\begin{longtable}[]{@{}llll@{}}
\toprule
\endhead
100 metres & 10.62 & Florence Griffith-Joyner & United States
(USA)\tabularnewline
200 metres & 21.34 & Florence Griffith-Joyner & United States
(USA)\tabularnewline
400 metres & 48.25 & Marie-José Pérec & France (FRA)\tabularnewline
800 metres & 1:53.43 & Nadezhda Olizarenko & Soviet Union
(URS)\tabularnewline
1,500 metres & 3:53.96 & Paula Ivan & Romania (ROU)\tabularnewline
5,000 metres & 14:26.17 & Vivian Cheruiyot & Kenya (KEN)\tabularnewline
10,000 metres & 29:17.45 & Almaz Ayana & Ethiopia (ETH)\tabularnewline
Marathon & 2:23:07 & Tiki Gelana & Ethiopia (ETH)\tabularnewline
\bottomrule
\end{longtable}

    \hypertarget{factor-analysis}{%
\subsection{5 - Factor Analysis}\label{factor-analysis}}

    The correlation matrix below is from the measurement of skeletal
features of white leghorn fowl (Dunn, Storrs Agricultural Experimental
Station Bulletin, 52, 1928). Where

X1 = Skull length X2 = Skull breadth X3 = Femur length X4 = Tibia length
X5 = Humerus length X6 = Ulna length

Using the \textbf{maximum likelihood procedure} the following estimated
factor loadings were extracted:

\begin{longtable}[]{@{}llllll@{}}
\toprule
Skeletal Feature & Variable & Estimated- & -Loadings & Varimax- &
-rotated-loadings\tabularnewline
\midrule
\endhead
& & F1 & F2 & F1* & F2*\tabularnewline
Skull length & 1 & 0.602 & 0.200 & 0.484 & 0.411\tabularnewline
Skull breadth & 2 & 0.467 & 0.154 & 0.375 & 0.319\tabularnewline
Femur length & 3 & 0.926 & 0.143 & 0.603 & 0.717\tabularnewline
Tibia length & 4 & 1.000 & 0.000 & 0.519 & 0.855\tabularnewline
Humerus length & 5 & 0.874 & 0.476 & 0.861 & 0.499\tabularnewline
Ulna length & 6 & 0.894 & 0.327 & 0.744 & 0.594\tabularnewline
\bottomrule
\end{longtable}

    \begin{longtable}[]{@{}llllllllll@{}}
\toprule
\begin{minipage}[b]{0.07\columnwidth}\raggedright
Skeletal Feature\strut
\end{minipage} & \begin{minipage}[b]{0.07\columnwidth}\raggedright
Variable\strut
\end{minipage} & \begin{minipage}[b]{0.07\columnwidth}\raggedright
Estimated-\strut
\end{minipage} & \begin{minipage}[b]{0.07\columnwidth}\raggedright
-Loadings\strut
\end{minipage} & \begin{minipage}[b]{0.07\columnwidth}\raggedright
\strut
\end{minipage} & \begin{minipage}[b]{0.07\columnwidth}\raggedright
\strut
\end{minipage} & \begin{minipage}[b]{0.07\columnwidth}\raggedright
Varimax-\strut
\end{minipage} & \begin{minipage}[b]{0.07\columnwidth}\raggedright
-rotated-loadings\strut
\end{minipage} & \begin{minipage}[b]{0.07\columnwidth}\raggedright
\strut
\end{minipage} & \begin{minipage}[b]{0.07\columnwidth}\raggedright
\strut
\end{minipage}\tabularnewline
\midrule
\endhead
\begin{minipage}[t]{0.07\columnwidth}\raggedright
\strut
\end{minipage} & \begin{minipage}[t]{0.07\columnwidth}\raggedright
\strut
\end{minipage} & \begin{minipage}[t]{0.07\columnwidth}\raggedright
F1\strut
\end{minipage} & \begin{minipage}[t]{0.07\columnwidth}\raggedright
F1 Proportions\strut
\end{minipage} & \begin{minipage}[t]{0.07\columnwidth}\raggedright
F2\strut
\end{minipage} & \begin{minipage}[t]{0.07\columnwidth}\raggedright
F2 Proportions\strut
\end{minipage} & \begin{minipage}[t]{0.07\columnwidth}\raggedright
F1*\strut
\end{minipage} & \begin{minipage}[t]{0.07\columnwidth}\raggedright
F1*Proportions\strut
\end{minipage} & \begin{minipage}[t]{0.07\columnwidth}\raggedright
F2*\strut
\end{minipage} & \begin{minipage}[t]{0.07\columnwidth}\raggedright
F2*Proportions\strut
\end{minipage}\tabularnewline
\begin{minipage}[t]{0.07\columnwidth}\raggedright
Skull length\strut
\end{minipage} & \begin{minipage}[t]{0.07\columnwidth}\raggedright
1\strut
\end{minipage} & \begin{minipage}[t]{0.07\columnwidth}\raggedright
0.602\strut
\end{minipage} & \begin{minipage}[t]{0.07\columnwidth}\raggedright
0.1263909\strut
\end{minipage} & \begin{minipage}[t]{0.07\columnwidth}\raggedright
0.200\strut
\end{minipage} & \begin{minipage}[t]{0.07\columnwidth}\raggedright
0.0419903\strut
\end{minipage} & \begin{minipage}[t]{0.07\columnwidth}\raggedright
0.484\strut
\end{minipage} & \begin{minipage}[t]{0.07\columnwidth}\raggedright
0.134969325\strut
\end{minipage} & \begin{minipage}[t]{0.07\columnwidth}\raggedright
0.411\strut
\end{minipage} & \begin{minipage}[t]{0.07\columnwidth}\raggedright
0.121060383\strut
\end{minipage}\tabularnewline
\begin{minipage}[t]{0.07\columnwidth}\raggedright
Skull breadth\strut
\end{minipage} & \begin{minipage}[t]{0.07\columnwidth}\raggedright
2\strut
\end{minipage} & \begin{minipage}[t]{0.07\columnwidth}\raggedright
0.467\strut
\end{minipage} & \begin{minipage}[t]{0.07\columnwidth}\raggedright
0.0980474\strut
\end{minipage} & \begin{minipage}[t]{0.07\columnwidth}\raggedright
0.154\strut
\end{minipage} & \begin{minipage}[t]{0.07\columnwidth}\raggedright
0.0323326\strut
\end{minipage} & \begin{minipage}[t]{0.07\columnwidth}\raggedright
0.375\strut
\end{minipage} & \begin{minipage}[t]{0.07\columnwidth}\raggedright
0.104573341\strut
\end{minipage} & \begin{minipage}[t]{0.07\columnwidth}\raggedright
0.319\strut
\end{minipage} & \begin{minipage}[t]{0.07\columnwidth}\raggedright
0.093961708\strut
\end{minipage}\tabularnewline
\begin{minipage}[t]{0.07\columnwidth}\raggedright
Femur length\strut
\end{minipage} & \begin{minipage}[t]{0.07\columnwidth}\raggedright
3\strut
\end{minipage} & \begin{minipage}[t]{0.07\columnwidth}\raggedright
0.926\strut
\end{minipage} & \begin{minipage}[t]{0.07\columnwidth}\raggedright
0.1944153\strut
\end{minipage} & \begin{minipage}[t]{0.07\columnwidth}\raggedright
0.143\strut
\end{minipage} & \begin{minipage}[t]{0.07\columnwidth}\raggedright
0.0300231\strut
\end{minipage} & \begin{minipage}[t]{0.07\columnwidth}\raggedright
0.603\strut
\end{minipage} & \begin{minipage}[t]{0.07\columnwidth}\raggedright
0.168153932\strut
\end{minipage} & \begin{minipage}[t]{0.07\columnwidth}\raggedright
0.717\strut
\end{minipage} & \begin{minipage}[t]{0.07\columnwidth}\raggedright
0.211192931\strut
\end{minipage}\tabularnewline
\begin{minipage}[t]{0.07\columnwidth}\raggedright
Tibia length\strut
\end{minipage} & \begin{minipage}[t]{0.07\columnwidth}\raggedright
4\strut
\end{minipage} & \begin{minipage}[t]{0.07\columnwidth}\raggedright
1.000\strut
\end{minipage} & \begin{minipage}[t]{0.07\columnwidth}\raggedright
0.2099517\strut
\end{minipage} & \begin{minipage}[t]{0.07\columnwidth}\raggedright
0.000\strut
\end{minipage} & \begin{minipage}[t]{0.07\columnwidth}\raggedright
0\strut
\end{minipage} & \begin{minipage}[t]{0.07\columnwidth}\raggedright
0.519\strut
\end{minipage} & \begin{minipage}[t]{0.07\columnwidth}\raggedright
0.144729504\strut
\end{minipage} & \begin{minipage}[t]{0.07\columnwidth}\raggedright
0.855\strut
\end{minipage} & \begin{minipage}[t]{0.07\columnwidth}\raggedright
0.251840943\strut
\end{minipage}\tabularnewline
\begin{minipage}[t]{0.07\columnwidth}\raggedright
Humerus length\strut
\end{minipage} & \begin{minipage}[t]{0.07\columnwidth}\raggedright
5\strut
\end{minipage} & \begin{minipage}[t]{0.07\columnwidth}\raggedright
0.874\strut
\end{minipage} & \begin{minipage}[t]{0.07\columnwidth}\raggedright
0.1834978\strut
\end{minipage} & \begin{minipage}[t]{0.07\columnwidth}\raggedright
0.476\strut
\end{minipage} & \begin{minipage}[t]{0.07\columnwidth}\raggedright
0.099937\strut
\end{minipage} & \begin{minipage}[t]{0.07\columnwidth}\raggedright
0.861\strut
\end{minipage} & \begin{minipage}[t]{0.07\columnwidth}\raggedright
0.24010039\strut
\end{minipage} & \begin{minipage}[t]{0.07\columnwidth}\raggedright
0.499\strut
\end{minipage} & \begin{minipage}[t]{0.07\columnwidth}\raggedright
0.146980854\strut
\end{minipage}\tabularnewline
\begin{minipage}[t]{0.07\columnwidth}\raggedright
Ulna length\strut
\end{minipage} & \begin{minipage}[t]{0.07\columnwidth}\raggedright
6\strut
\end{minipage} & \begin{minipage}[t]{0.07\columnwidth}\raggedright
0.894\strut
\end{minipage} & \begin{minipage}[t]{0.07\columnwidth}\raggedright
0.1876968\strut
\end{minipage} & \begin{minipage}[t]{0.07\columnwidth}\raggedright
0.327\strut
\end{minipage} & \begin{minipage}[t]{0.07\columnwidth}\raggedright
0.0686542\strut
\end{minipage} & \begin{minipage}[t]{0.07\columnwidth}\raggedright
0.744\strut
\end{minipage} & \begin{minipage}[t]{0.07\columnwidth}\raggedright
0.207473508\strut
\end{minipage} & \begin{minipage}[t]{0.07\columnwidth}\raggedright
0.594\strut
\end{minipage} & \begin{minipage}[t]{0.07\columnwidth}\raggedright
0.174963181\strut
\end{minipage}\tabularnewline
\begin{minipage}[t]{0.07\columnwidth}\raggedright
\textbf{Totals}\strut
\end{minipage} & \begin{minipage}[t]{0.07\columnwidth}\raggedright
\strut
\end{minipage} & \begin{minipage}[t]{0.07\columnwidth}\raggedright
4.763\strut
\end{minipage} & \begin{minipage}[t]{0.07\columnwidth}\raggedright
\strut
\end{minipage} & \begin{minipage}[t]{0.07\columnwidth}\raggedright
1.3\strut
\end{minipage} & \begin{minipage}[t]{0.07\columnwidth}\raggedright
\strut
\end{minipage} & \begin{minipage}[t]{0.07\columnwidth}\raggedright
3.586\strut
\end{minipage} & \begin{minipage}[t]{0.07\columnwidth}\raggedright
\strut
\end{minipage} & \begin{minipage}[t]{0.07\columnwidth}\raggedright
3.395\strut
\end{minipage} & \begin{minipage}[t]{0.07\columnwidth}\raggedright
\strut
\end{minipage}\tabularnewline
\bottomrule
\end{longtable}

    \begin{Verbatim}[commandchars=\\\{\}]
{\color{incolor}In [{\color{incolor}28}]:} \PY{k+kr}{proc iml;}
         fowlCorr = \PYZob{}
         \PY{l+m}{1.000} \PY{l+m}{0.505} \PY{l+m}{0.569} \PY{l+m}{0.602} \PY{l+m}{0.621} \PY{l+m}{0.603},
         \PY{l+m}{0.505} \PY{l+m}{1.000} \PY{l+m}{0.422} \PY{l+m}{0.467} \PY{l+m}{0.482} \PY{l+m}{0.450},
         \PY{l+m}{0.569} \PY{l+m}{0.422} \PY{l+m}{1.000} \PY{l+m}{0.926} \PY{l+m}{0.877} \PY{l+m}{0.878},
         \PY{l+m}{0.602} \PY{l+m}{0.467} \PY{l+m}{0.926} \PY{l+m}{1.000} \PY{l+m}{0.874} \PY{l+m}{0.894},
         \PY{l+m}{0.621} \PY{l+m}{0.482} \PY{l+m}{0.877} \PY{l+m}{0.874} \PY{l+m}{1.000} \PY{l+m}{0.937},
         \PY{l+m}{0.603} \PY{l+m}{0.450} \PY{l+m}{0.878} \PY{l+m}{0.894} \PY{l+m}{0.937} \PY{l+m}{1.000}
         \PYZcb{};
\end{Verbatim}


\begin{Verbatim}[commandchars=\\\{\}]
{\color{outcolor}Out[{\color{outcolor}28}]:} <IPython.core.display.HTML object>
\end{Verbatim}
            
    \begin{Verbatim}[commandchars=\\\{\}]
{\color{incolor}In [{\color{incolor}29}]:} \PY{c+cm}{/* data fowl (type=cov);}
         \PY{c+cm}{input \PYZus{}type\PYZus{}\PYZdl{} x1\PYZhy{}x6;}
         \PY{c+cm}{datalines;}
         \PY{c+cm}{corr 1.000 0.505 0.569 0.602 0.621 0.603}
         \PY{c+cm}{corr 0.505 1.000 0.422 0.467 0.482 0.450}
         \PY{c+cm}{corr 0.569 0.422 1.000 0.926 0.877 0.878}
         \PY{c+cm}{corr 0.602 0.467 0.926 1.000 0.874 0.894}
         \PY{c+cm}{corr 0.621 0.482 0.877 0.874 1.000 0.937}
         \PY{c+cm}{corr 0.603 0.450 0.878 0.894 0.937 1.000}
         \PY{c+cm}{;}
         
         \PY{c+cm}{proc factor data=fowl method=principal corr p=80;}
         
         \PY{c+cm}{/* proc factor data=fowl method=principal cov rotate=varimax; */}
         \PY{k+kr}{run;} */
\end{Verbatim}


\begin{Verbatim}[commandchars=\\\{\}]
{\color{outcolor}Out[{\color{outcolor}29}]:} <IPython.core.display.HTML object>
\end{Verbatim}
            
    \hypertarget{d-using-the-unrotated-estimated-factor-loadings-obtain-the-maximum-likelihood-estimates-of-the-following}{%
\subsubsection{\texorpdfstring{d) Using the \textbf{unrotated estimated
factor} loadings, obtain the maximum likelihood estimates of the
following:}{d) Using the unrotated estimated factor loadings, obtain the maximum likelihood estimates of the following:}}\label{d-using-the-unrotated-estimated-factor-loadings-obtain-the-maximum-likelihood-estimates-of-the-following}}

    \hypertarget{i.-the-specific-variances.}{%
\paragraph{\texorpdfstring{i. The specific variances. \$\hat { \psi  }
\$}{i. The specific variances. \$ \$}}\label{i.-the-specific-variances.}}

    \begin{Verbatim}[commandchars=\\\{\}]
{\color{incolor}In [{\color{incolor}30}]:} \PY{k+kr}{proc iml;}
         fowlCorr = \PYZob{}
         \PY{l+m}{1.000} \PY{l+m}{0.505} \PY{l+m}{0.569} \PY{l+m}{0.602} \PY{l+m}{0.621} \PY{l+m}{0.603},
         \PY{l+m}{0.505} \PY{l+m}{1.000} \PY{l+m}{0.422} \PY{l+m}{0.467} \PY{l+m}{0.482} \PY{l+m}{0.450},
         \PY{l+m}{0.569} \PY{l+m}{0.422} \PY{l+m}{1.000} \PY{l+m}{0.926} \PY{l+m}{0.877} \PY{l+m}{0.878},
         \PY{l+m}{0.602} \PY{l+m}{0.467} \PY{l+m}{0.926} \PY{l+m}{1.000} \PY{l+m}{0.874} \PY{l+m}{0.894},
         \PY{l+m}{0.621} \PY{l+m}{0.482} \PY{l+m}{0.877} \PY{l+m}{0.874} \PY{l+m}{1.000} \PY{l+m}{0.937},
         \PY{l+m}{0.603} \PY{l+m}{0.450} \PY{l+m}{0.878} \PY{l+m}{0.894} \PY{l+m}{0.937} \PY{l+m}{1.000}
         \PYZcb{};
         estimatedFactorLoading = \PYZob{}
         \PY{l+m}{0.602} \PY{l+m}{0.200},
         \PY{l+m}{0.467} \PY{l+m}{0.154},
         \PY{l+m}{0.926} \PY{l+m}{0.143},
         \PY{l+m}{1.000} \PY{l+m}{0.000},
         \PY{l+m}{0.874} \PY{l+m}{0.476},
         \PY{l+m}{0.894} \PY{l+m}{0.327}
         \PYZcb{};
         estimatedFactorLoadingT = t(estimatedFactorLoading);
         LLt = estimatedFactorLoading * estimatedFactorLoadingT;
         
         psiWithoutZeros = fowlCorr \PYZhy{} LLt;
         psi = diag(psiWithoutZeros);
         psiFowlEstimated = vecdiag(psi);
         fowlCorrEstimated = LLt + psi;
         redisualMatrixEstimated = fowlCorr \PYZhy{} LLt \PYZhy{} psi;
         
         communalitiesFowlEstimated = vecDiag(\PY{l+m}{1} \PYZhy{} psi);
         print psiFowlEstimated;
         print fowlCorrEstimated;
         print fowlCorr;
\end{Verbatim}


\begin{Verbatim}[commandchars=\\\{\}]
{\color{outcolor}Out[{\color{outcolor}30}]:} <IPython.core.display.HTML object>
\end{Verbatim}
            
    \hypertarget{ii.-the-communalities.-h-i-2-1---i}{%
\paragraph{\texorpdfstring{ii. The communalities. \$ \{ h \}\emph{\{ i
\}\^{}\{ 2 \} = 1 -\{ \psi  \}}\{ i
\}\$}{ii. The communalities. \$ \{ h \}\{ i \}\^{}\{ 2 \} = 1 -\{ \}\{ i \}\$}}\label{ii.-the-communalities.-h-i-2-1---i}}

    \begin{Verbatim}[commandchars=\\\{\}]
{\color{incolor}In [{\color{incolor}31}]:} print communalitiesFowlEstimated;
\end{Verbatim}


\begin{Verbatim}[commandchars=\\\{\}]
{\color{outcolor}Out[{\color{outcolor}31}]:} <IPython.core.display.HTML object>
\end{Verbatim}
            
    \hypertarget{iii.-the-proportion-of-variance-explained-by-each-factor.}{%
\paragraph{iii. The proportion of variance explained by each
factor.}\label{iii.-the-proportion-of-variance-explained-by-each-factor.}}

    \textbf{Estimated-Loadings}

\begin{longtable}[]{@{}llllll@{}}
\toprule
Skeletal Feature & Variable & & & &\tabularnewline
\midrule
\endhead
& & F1 & F1 Proportions & F2 & F2 Proportions\tabularnewline
Skull length & x1 & 0.602 & 0.1263909 & 0.200 & 0.0419903\tabularnewline
Skull breadth & x2 & 0.467 & 0.0980474 & 0.154 &
0.0323326\tabularnewline
Femur length & x3 & 0.926 & 0.1944153 & 0.143 & 0.0300231\tabularnewline
Tibia length & x4 & 1.000 & 0.2099517 & 0.000 & 0\tabularnewline
Humerus length & x5 & 0.874 & 0.1834978 & 0.476 &
0.099937\tabularnewline
Ulna length & x6 & 0.894 & 0.1876968 & 0.327 & 0.0686542\tabularnewline
\textbf{Totals} & & \textbf{4.763} & & \textbf{1.300} &\tabularnewline
\bottomrule
\end{longtable}

    \hypertarget{iv.-the-residual-matrix-2.5-marks}{%
\paragraph{iv. The residual matrix (2.5
marks)}\label{iv.-the-residual-matrix-2.5-marks}}

    \$R-\widehat { L } \widehat { L } -\hat { \psi  } \$

    \begin{Verbatim}[commandchars=\\\{\}]
{\color{incolor}In [{\color{incolor}32}]:} print redisualMatrixEstimated;
\end{Verbatim}


\begin{Verbatim}[commandchars=\\\{\}]
{\color{outcolor}Out[{\color{outcolor}32}]:} <IPython.core.display.HTML object>
\end{Verbatim}
            
    \hypertarget{e-using-the-varimax-rotated-estimated-factor-loadings-obtain-the-maximum-likelihood-estimates-of-the-following}{%
\paragraph{e) Using the varimax rotated estimated factor loadings,
obtain the maximum likelihood estimates of the
following:}\label{e-using-the-varimax-rotated-estimated-factor-loadings-obtain-the-maximum-likelihood-estimates-of-the-following}}

\hypertarget{i.-the-specific-variances.}{%
\paragraph{i. The specific
variances.}\label{i.-the-specific-variances.}}

    \begin{Verbatim}[commandchars=\\\{\}]
{\color{incolor}In [{\color{incolor}33}]:} \PY{k+kr}{proc iml;}
         fowlCorr = \PYZob{}
         \PY{l+m}{1.000} \PY{l+m}{0.505} \PY{l+m}{0.569} \PY{l+m}{0.602} \PY{l+m}{0.621} \PY{l+m}{0.603},
         \PY{l+m}{0.505} \PY{l+m}{1.000} \PY{l+m}{0.422} \PY{l+m}{0.467} \PY{l+m}{0.482} \PY{l+m}{0.450},
         \PY{l+m}{0.569} \PY{l+m}{0.422} \PY{l+m}{1.000} \PY{l+m}{0.926} \PY{l+m}{0.877} \PY{l+m}{0.878},
         \PY{l+m}{0.602} \PY{l+m}{0.467} \PY{l+m}{0.926} \PY{l+m}{1.000} \PY{l+m}{0.874} \PY{l+m}{0.894},
         \PY{l+m}{0.621} \PY{l+m}{0.482} \PY{l+m}{0.877} \PY{l+m}{0.874} \PY{l+m}{1.000} \PY{l+m}{0.937},
         \PY{l+m}{0.603} \PY{l+m}{0.450} \PY{l+m}{0.878} \PY{l+m}{0.894} \PY{l+m}{0.937} \PY{l+m}{1.000}
         \PYZcb{};
         
         VarimaxRotated = \PYZob{}
         \PY{l+m}{0.484} \PY{l+m}{0.411},
         \PY{l+m}{0.375} \PY{l+m}{0.319},
         \PY{l+m}{0.603} \PY{l+m}{0.717},
         \PY{l+m}{0.519} \PY{l+m}{0.855},
         \PY{l+m}{0.861} \PY{l+m}{0.499},
         \PY{l+m}{0.744} \PY{l+m}{0.594}
         \PYZcb{};
         
         redisualMatrixEstimated = \PYZob{}
         \PY{l+m}{0} \PY{l+m}{0.193066} \PYZhy{}\PY{l+m}{0.017052} \PY{l+m}{0} \PYZhy{}\PY{l+m}{0.000348} \PYZhy{}\PY{l+m}{0.000588},
         \PY{l+m}{0.193066} \PY{l+m}{0} \PYZhy{}\PY{l+m}{0.032464} \PY{l+m}{0} \PY{l+m}{0.000538} \PYZhy{}\PY{l+m}{0.017856},
         \PYZhy{}\PY{l+m}{0.017052} \PYZhy{}\PY{l+m}{0.032464} \PY{l+m}{0} \PY{l+m}{0} \PYZhy{}\PY{l+m}{0.000392} \PY{l+m}{0.003395},
         \PY{l+m}{0} \PY{l+m}{0} \PY{l+m}{0} \PY{l+m}{0} \PY{l+m}{0} \PY{l+m}{0},
         \PYZhy{}\PY{l+m}{0.000348} \PY{l+m}{0.000538} \PYZhy{}\PY{l+m}{0.000392} \PY{l+m}{0} \PY{l+m}{0} \PYZhy{}\PY{l+m}{8E\PYZhy{}6},
         \PYZhy{}\PY{l+m}{0.000588} \PYZhy{}\PY{l+m}{0.017856} \PY{l+m}{0.003395} \PY{l+m}{0} \PYZhy{}\PY{l+m}{8E\PYZhy{}6} \PY{l+m}{0} 
         \PYZcb{};
         
         VarimaxRotatedT = t(VarimaxRotated);
         LLt = VarimaxRotated * VarimaxRotatedT;
         
         psiWithoutZeros = fowlCorr \PYZhy{} LLt;
         psi = diag(psiWithoutZeros);
         psiFowlVarimax = vecdiag(psi);
         fowlCorrSample = LLt + psi;
         redisualMatrixRotated = fowlCorr \PYZhy{} LLt \PYZhy{} psi;
         
         residualMatrixDiff = redisualMatrixEstimated \PYZhy{} redisualMatrixRotated;
         
         communalitiesFowlVarimax = vecDiag(\PY{l+m}{1} \PYZhy{} psi);
         print psiFowlVarimax;
         print fowlCorrSample;
         print fowlCorr;
\end{Verbatim}


\begin{Verbatim}[commandchars=\\\{\}]
{\color{outcolor}Out[{\color{outcolor}33}]:} <IPython.core.display.HTML object>
\end{Verbatim}
            
    \hypertarget{ii.-the-communalities.}{%
\paragraph{ii. The communalities.}\label{ii.-the-communalities.}}

    \begin{Verbatim}[commandchars=\\\{\}]
{\color{incolor}In [{\color{incolor}34}]:} print communalitiesFowlVarimax;
\end{Verbatim}


\begin{Verbatim}[commandchars=\\\{\}]
{\color{outcolor}Out[{\color{outcolor}34}]:} <IPython.core.display.HTML object>
\end{Verbatim}
            
    \hypertarget{iii.-the-proportion-of-variance-explained-by-each-factor.}{%
\paragraph{iii. The proportion of variance explained by each
factor.}\label{iii.-the-proportion-of-variance-explained-by-each-factor.}}

    \hypertarget{varimax-rotated-loadings}{%
\paragraph{Varimax-rotated-loadings}\label{varimax-rotated-loadings}}

\begin{longtable}[]{@{}llllll@{}}
\toprule
\endhead
& & F1* & F1*Proportions & F2* & F2*Proportions\tabularnewline
Skull length & 1 & 0.484 & 0.134969325 & 0.411 &
0.121060383\tabularnewline
Skull breadth & 2 & 0.375 & 0.104573341 & 0.319 &
0.093961708\tabularnewline
Femur length & 3 & 0.603 & 0.168153932 & 0.717 &
0.211192931\tabularnewline
Tibia length & 4 & 0.519 & 0.144729504 & 0.855 &
0.251840943\tabularnewline
Humerus length & 5 & 0.861 & 0.24010039 & 0.499 &
0.146980854\tabularnewline
Ulna length & 6 & 0.744 & 0.207473508 & 0.594 &
0.174963181\tabularnewline
\textbf{Totals} & & 3.586 & & 3.395 &\tabularnewline
\bottomrule
\end{longtable}

    \hypertarget{iv.-theresidualmatrix-r----2.5-marks}{%
\paragraph{\texorpdfstring{iv. Theresidualmatrix \$R-\widehat { L }
\widehat { L } -\hat { \psi  } \$ (2.5
marks)}{iv. Theresidualmatrix \$R-  - \$ (2.5 marks)}}\label{iv.-theresidualmatrix-r----2.5-marks}}

    \begin{Verbatim}[commandchars=\\\{\}]
{\color{incolor}In [{\color{incolor}35}]:} print redisualMatrixRotated;
\end{Verbatim}


\begin{Verbatim}[commandchars=\\\{\}]
{\color{outcolor}Out[{\color{outcolor}35}]:} <IPython.core.display.HTML object>
\end{Verbatim}
            
    \hypertarget{f-comment-on-the-results-using-the-two-loading-methods-by-comparing-your-results-from-part-a-and-part-b-above.-2-marks}{%
\paragraph{f) Comment on the results using the two loading methods by
comparing your results from part a) and part b) above. (2
marks)}\label{f-comment-on-the-results-using-the-two-loading-methods-by-comparing-your-results-from-part-a-and-part-b-above.-2-marks}}

    Rotating the residual matrix

Everything appears to be quite similar, except for {[}2,1{]} in our
corrolation matrix:

\begin{itemize}
\tightlist
\item
  Estimated loading 0.311934
\item
  Varimax rotated 0.312609
\item
  Corrolation matrix 0.505
\end{itemize}

Which seems to be slightly different based on our sample.

    \begin{Verbatim}[commandchars=\\\{\}]
{\color{incolor}In [{\color{incolor}36}]:} print residualMatrixDiff;
\end{Verbatim}


\begin{Verbatim}[commandchars=\\\{\}]
{\color{outcolor}Out[{\color{outcolor}36}]:} <IPython.core.display.HTML object>
\end{Verbatim}
            
    It's not obvious with the estimated loadings what is explained by F1,
and F2 in the estimated loading

Ie. F1 =\textgreater{} mostly explained by Tibia, Femur, Ulna, Humerous,
each \textasciitilde{}20\% F2 =\textgreater{} mostly explained by Ulna
(0.07\textasciitilde{})

Rotated is much easier to understand:

F1 =\textgreater{} Mostly explained by Humerus (24\%), Ulna (20.7\%),
Femur (17\%) F2 =\textgreater{} Mostly explained by Tibia (25\%),
Femur(21\%), Ulna (17\%)

There is no part a) or part b)

Not rotated very far rotating moves this away from psi of 0 and
communality of 1 So psi{[}4{]} for Varimax gets rotated to -0.000386

    


    % Add a bibliography block to the postdoc
    
    
    
    \end{document}
